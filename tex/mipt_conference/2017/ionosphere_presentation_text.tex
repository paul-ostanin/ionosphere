\documentclass[2pt, a4paper, fleqn]{extarticle}

\usepackage[T2A]{fontenc}
\usepackage[utf8]{inputenc}
\usepackage[english,russian]{babel}
\usepackage{url}
%\usepackage{pscyr}

%\renewcommand{\rmdefault}{ftm}
\usepackage{setspace}
\onehalfspacing

\usepackage{changepage}
\usepackage{indentfirst} %первый абзац
%%\usepackage{moreverb}
\usepackage[noend]{algorithmic}
\usepackage{amssymb, amsmath, multicol,amsthm}
%%
\usepackage{enumitem, multicol}
\usepackage{titleps,lipsum}
%%
\usepackage{mathrsfs}
\usepackage{verbatim}
\usepackage{pb-diagram}
\usepackage{graphicx}
\graphicspath{ {images/} }
\usepackage{wrapfig}
\usepackage{xcolor}
\definecolor{new}{RGB}{255,184,92}
\definecolor{news}{RGB}{112,112,112}
\usepackage{wallpaper}
\usepackage{float}
\usepackage{hyperref}
\hypersetup{
%colorlinks=true,%
%linkcolor=news,%
linkbordercolor=new,
}



\usepackage{geometry}
\geometry{top=1cm,bottom=2cm,left=1cm,right=1cm}

%\flushbottom
%\ruggedbottom

\binoppenalty=5000
\parindent=0pt

\newcommand{\EDS}{\ensuremath{\mathscr{E}}}
\newcommand*{\hm}[1]{#1\nobreak\discretionary{}%
{\hbox{$\mathsurround=0pt #1$}}{}}
\newcommand{\divisible}{\mathop{\raisebox{-2pt}{\vdots}}}
\renewcommand{\theequation}{\arabic{equation}}
\def\hm#1{#1\nobreak\discretionary{}{\hbox{$#1$}}{}}
\newcommand{\bbskip}{\bigskip \bigskip}



%%\DeclareMathOperator{\tg}{tg}
%%\DeclareMathOperator{\ctg}{ctg}

\let\leq\leqslant
\let\geq\geqslant



% Remove brackets from numbering in List of References
\makeatletter
\renewcommand{\@biblabel}[1]{\quad#1.}
\makeatother

\begin{document}

{\bf Слайд 2: Введение и постановка}
В работе рассматривается задача по построению динамической модели Земной ионосферы и последующего включения этой модели в качестве вычислительного блока в совместную модель верхней атмосферы. 

При разработке используются приближения: рассмотрение только F слоя, динамическое преобладание амбиполярной диффузии, одноионная постановка (только $O+$), дипольное магнитное поле Земли, приближение совпадения географических и магнитных полюсов, квазинейтральность плазмы.

В этих предположениях уравнение неразрывности для электронной плотности можно записать в следующем виде.

Уравнение имеет ряд особенностей: неотрицательность решения, наличие закона сохранения массы в отсутствие источников, а также существенно различные характерные времена различных физических процессов, что обуславливает сильную жесткость.
Кроме этого рассматриваемое уравнение при неизменных $P$ и $k$ имеет стационарное решение, что и будет рассмотрено в презентации.

\medskip

{\bf Слайд 3: уравнение в сфер. коорд. в приближении тонкого сферич. слоя.}

Полное трёхмерное уравнение в сферических координатах представлено на слайде.

\medskip

{\bf Слайд 4: метод расщепления.}

Основным используемым в задаче методом является метод расщепления, причем расщепление происходит как по физическим процессам, так и по геометрическим переменным. Так, на первом шаге решается задача, отвечающая фотохимическим процессам фотоионизации и рекомбинации, плюс диффузия вдоль оси $z$. На втором шаге решается задача, отвечающая диффузии вдоль координаты $\varphi$. На третьем шаге расщепления решается задача, описывающая трёхмерный перенос.

Сразу оговоримся по поводу используемых схем: слагаемые в уравнениях для первого и второго шага имеют один и тот же вид: в обоих есть диффузионная составляющая, перенос, а также смешанная производная. Поэтому схемы рассмотрим единые для обоих шагов.

\medskip

{\bf Слайд 5: Используемые схемы.}

Для диффузионного и дивергентного слагаемых используются стандартные аппроксимации центральными разностями. Интерес вызывает аппроксимация смешанной производной.

При положительном значении $B(\varphi)$ (в верхнем полушарии) с помощью центральных разностей аппроксимируется производная в центрах квадратов I и IV, а в южном полушарии~---~в центрах II и III квадратов.

Итоговая аппроксимация представляет собой полусумму вычисленных в центрах квадратов смешанных производных в соответствии со знаком $B$.



\medskip

{\bf Слайд 6: околополюсные точки}
Отметим особенность, связанную с аппроксимацией по $\varphi$ вблизи полюсов. Выбранная сетка по $\varphi$ равномерна и имеет шаг $1^\circ$, узлы попадают в полуцелые значения градусов. Используемые схемы аппроксимируют каждое слагаемое со вторым порядком, но несмотря на это общий порядок аппроксимации по $\varphi$ первый: вблизи границ деление на косинус уменьшает порядок аппроксимации на единицу ($\cos\varphi$~---~величина порядка шага сетки в первой степени).

Помимо этого при использовании выбранной аппроксимации смешанной производной возникает необходимость для вычисления смешанной производной в узле $j$ использовать информацию с соседних узлов. При этом около полюсов необходимо использовать информацию из точек, находящихся за полюсами, формально, вне расчетной области. Для решения этой проблемы точки по разные стороны от полюса вблизи него отождествляются.

\medskip

{\bf Слайд 7: Рассматриваемые модели}

Прежде всего обратимся к одномерной задаче для $z$-диффузии без проекций. Это простейшая модель, не учитывающая широтную зависимость, а магнитное поле в ней считается направленным вертикально.

Существенно усложненная задача~---~полный первый шаг расщепления, хорошо описывающий распределение в средних широтах, но не описывающий физику на экваторе, где поле горизонтально.

Наконец, двумерная модель построена по двум шагам расщепления, и уточняет первую. Перейдем к сравнению результатов расчетов по этим трём моделям.
\medskip

{\bf Слайд 8. Результаты расчетов.}


Вблизи полюсов не важно, по какой схеме считать, в средних широтах проявляются различия одномерной схемы и схем первого и двух шагов, а вблизи экватора различия существенны для всех трёх постановок.
Можно сравнить постановки по первому и по двум шагам по поверхностям $n = n(z, \varphi)$ - стационарным распределениям на всех высотах и по всем широтам в рассматриваемой области.
\end{document}



