\documentclass[2pt, a4paper, fleqn]{extarticle}

\usepackage[T2A]{fontenc}
\usepackage[utf8]{inputenc}
\usepackage[english,russian]{babel}
\usepackage{url}
%\usepackage{pscyr}

%\renewcommand{\rmdefault}{ftm}
\usepackage{setspace}
\onehalfspacing

\usepackage{changepage}
\usepackage{indentfirst} %первый абзац
%%\usepackage{moreverb}
\usepackage[noend]{algorithmic}
\usepackage{amssymb, amsmath, multicol,amsthm}
%%
\usepackage{enumitem, multicol}
\usepackage{titleps,lipsum}
%%
\usepackage{mathrsfs}
\usepackage{verbatim}
\usepackage{pb-diagram}
\usepackage{graphicx}
\graphicspath{ {images/} }
\usepackage{wrapfig}
\usepackage{xcolor}
\definecolor{new}{RGB}{255,184,92}
\definecolor{news}{RGB}{112,112,112}
\usepackage{wallpaper}
\usepackage{float}
\usepackage{hyperref}
\hypersetup{
%colorlinks=true,%
%linkcolor=news,%
linkbordercolor=new,
}



\usepackage{geometry}
\geometry{top=1cm,bottom=2cm,left=1cm,right=1cm}

%\flushbottom
%\ruggedbottom

\binoppenalty=5000
\parindent=0pt

\newcommand{\EDS}{\ensuremath{\mathscr{E}}}
\newcommand*{\hm}[1]{#1\nobreak\discretionary{}%
{\hbox{$\mathsurround=0pt #1$}}{}}
\newcommand{\divisible}{\mathop{\raisebox{-2pt}{\vdots}}}
\renewcommand{\theequation}{\arabic{equation}}
\def\hm#1{#1\nobreak\discretionary{}{\hbox{$#1$}}{}}
\newcommand{\bbskip}{\bigskip \bigskip}



%%\DeclareMathOperator{\tg}{tg}
%%\DeclareMathOperator{\ctg}{ctg}

\let\leq\leqslant
\let\geq\geqslant



% Remove brackets from numbering in List of References
\makeatletter
\renewcommand{\@biblabel}[1]{\quad#1.}
\makeatother

\begin{document}

Рассмотрим для вычисления $\dfrac{1}{n} \dfrac{\partial n}{\partial z}$ следующие две схемы:

\begin{itemize}
\item $\dfrac{2}{n_{i, j+1}+n_{i, j-1}}\dfrac{n_{i, j+1}-n_{i, j}}{\Delta\varphi}$
\item $\dfrac{2}{n_{i, j+1}+n_{i, j-1}}\dfrac{n_{i, j+1}-n_{i, j-1}}{2\Delta\varphi}$
\end{itemize}

Первая схема~---~схема направленных разностей, обладает численной диффузией, а вторая~---~схема центральных разностей~---~не обладает ей. После замены второй формулы на первую в схему была внесена численная диффузия, которую можно оценить из разности этих аппроксимаций: $$\dfrac{2}{n_{i, j+1}+n_{i, j-1}}\dfrac{n_{i, j+1}-2n_{i, j}+n_{i, j-1}}{\Delta\varphi^2} \cdot\dfrac{h}{2} \approx\dfrac{1}{n} \cdot \dfrac{h}{2}\cdot \dfrac{\partial^2 n}{\partial \varphi^2}$$

В уравнении для первого шага рассматривается $\dfrac{\partial}{\partial z}\left[n \dfrac{1}{n} \dfrac{\partial n}{\partial z}\right]$. Тогда в результате в схему вносится слагаемое, оценивающееся как $$\dfrac{\partial}{\partial z}\left[\dfrac{h}{2}\dfrac{\partial^2 n}{\partial \varphi^2}\right]$$
Это слагаемое вносит третью смешанную производную в первое дифференциальное приближение схемы (дважды по $\varphi$ и один раз по $z$).
\end{document}



