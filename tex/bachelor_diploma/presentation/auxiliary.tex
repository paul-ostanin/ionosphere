
\newpage

\textit{Шпаргалка.} Дополнение (если спросят): формулы для суточного хода.
Для моделирования суточного изменения вертикального профиля добавим зависимость от времени в слагаемое $P$, отвечающее фотоионизации.

Используем формулу $$P(z, t) =\begin{cases}
P_0(z)e^{\tau_0(z)(1-\sec\chi)}, |\chi|\leq\dfrac{\pi}{2}\\
0, |\chi|\geq\dfrac{\pi}{2}
\end{cases}$$

Здесь использованы следующие обозначения: $P_0(z)$~---~фотоионизация в дневное время, $\chi$~---~зенитный угол Солнца (угол между направлением на Солнце и нормалью к земной поверхности), $\tau_0(z)$~---~оптическая толщина, для вычисления которой используется формула $$\tau_0(z)=\sum_{i = N_2, O_2, O} \sigma_i^{abs}\left[\dfrac{R_0T_n}{M_i g}n_i(z)\right]= \dfrac{R_0T_n}{g}\left(\sigma_{N_2}^{abs}\dfrac{n_{N_2}(z)}{M_{N_2}}+\sigma_{O_2}^{abs}\dfrac{n_{O_2}(z)}{M_{O_2}}+\sigma_{O}^{abs}\dfrac{n_{O}(z)}{M_{O}}\right).$$

Константы $\sigma_i^{abs}$ для трёх типов нейтральных молекул известны и равны соответственно $\sigma_{N_2}^{abs}=1{,}5\cdot 10^{-17}$~см$^2$, $\sigma_{O_2}^{abs}=2\cdot 10^{-17}$~см$^2$, $\sigma_{O}^{abs}=1\cdot 10^{-17}$~см$^2$.

Характерные величины оптической толщины на различных высотах представлены в следующей таблице:

\smallskip

$$\begin{tabular}{|c|c|c|c|}
\hline
&$z_1=100$~км&$z_2=300$~км&$z_3=500$~км\\
\hline
$\tau_0$&$4\cdot 10^2$&$3\cdot 10^{-1}$&$2\cdot 10^{-4}$\\
\hline
\end{tabular}$$

\medskip

В предложенной формуле для фотоионизации время в качестве параметра входит лишь в зенитный угол. Кусочное задание функции $P(z, t)$ связано с приближением отсутствия фотоионизации в ночное время (Солнце не заходит за горизонт лишь при зенитных углах, не превосходящих $90^\circ$).

Зависимость зенитного угла от времени даётся следующими формулами: $$\cos\chi = \sin\varphi\cdot\sin\delta-\cos\varphi\cdot\cos\delta\cdot\cos\omega t$$

Здесь $\omega$~---~угловая скорость вращения Земли, $\varphi$~---~широта, а $\delta$~---~склонение Солнца, тангенс которого определяется формулой $$\tg\delta = \tg 23{,}5^\circ \cdot \sin\left(2\pi\cdot\dfrac{d-80}{365}\right),$$ где $d$~---~номер дня от начала года.

\textit{ Дополнение о гран. условиях}

\textit{ Нижнее граничное условие (условие Дирихле) аппроксимируется точно, а на верхней границе условие постоянства потока может быть записано несколькими способами. Для данной одномерной задачи используем две различных аппроксимации этого условия:}


\textit{В первом случае поток $\dfrac{\partial n}{\partial z}+\dfrac{u_N}{D_N}\cdot n_N=F$ аппроксимируется с помощью центральных разностей по пространству, что соответствует схеме $n_N-n_{N-1}+u_N/D_N\cdot h_N\cdot n_N = F\cdot h_N$}

\textit{Во втором случае для схемы центральных разностей запишем согласованную схему для верхнего граничного случая, получаемую с помощью интегрирования уравнения на $N$-ом шаге по пространству между двумя соседними полуцелыми узлами, а также учёта равенства потока на верхнем полуцелом узле заданной величине $F$: $h_{N+1/2}\dfrac{n^{j+1}-n^j}{\tau}= F - D_{N-1/2}\dfrac{n_N-n_{N-1}}{h_{N-1}}-\dfrac{1}{2}(u_{N-1}n_{N-1}^{j+1}+u_{N}n_{N}^{j+1})$
}
 
 Характерные величины на нескольких высотах представлены в следующей таблице: 

\smallskip

\begin{tabular}{|c|c|c|c|}
\hline
&$z_1=200$~км&$z_2=300$~км&$z_3=500$~км\\
\hline
$D$, см$^{2}\cdot$с$^{-1}$&$3{,}1\cdot 10^9$&$3{,}4\cdot 10^{10}$&$4{,}2\cdot 10^{12}$\\
\hline
$k$, с$^{-1}$&$5{,}2\cdot 10^{-3}$&$5{,}5\cdot 10^{-5}$&$1{,}3\cdot 10^{-8}$\\
\hline
$P_1$, см$^{-3}\cdot$с$^{-1}$&$1{,}5\cdot 10^3$&$1{,}2\cdot 10^{2}$&$1{,}3$\\
\hline
$u_\textrm{эфф}/D$, см$^{-1}$&$4{,}8\cdot 10^{-8}$&$4{,}5\cdot 10^{-8}$&$3{,}6\cdot 10^{-8}$\\
\hline
\end{tabular}

\medskip

