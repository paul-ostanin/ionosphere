\documentclass[14pt, a4paper, fleqn, twoside]{extreport}

\usepackage[T2A]{fontenc}
\usepackage[utf8]{inputenc}
\usepackage[english,russian]{babel}
\usepackage{url}
\usepackage{pscyr}

\renewcommand{\rmdefault}{ftm}
\usepackage{setspace}
\onehalfspacing

\usepackage{changepage}
\usepackage{indentfirst} %первый абзац
%%\usepackage{moreverb}
\usepackage[noend]{algorithmic}
\usepackage{amssymb, amsmath, multicol,amsthm}
%%
\usepackage{enumitem, multicol}
\usepackage{titleps,lipsum}
%%
\usepackage{mathrsfs}
\usepackage{verbatim}
\usepackage{pb-diagram}
\usepackage{graphicx}
\graphicspath{ {images/} }
\usepackage{wrapfig}
\usepackage{xcolor}
\definecolor{new}{RGB}{255,184,92}
\definecolor{news}{RGB}{112,112,112}
\usepackage{wallpaper}
\usepackage{hyperref}
\hypersetup{
%colorlinks=true,%
%linkcolor=news,%
linkbordercolor=new,
}


\usepackage{geometry}
\geometry{top=3.5cm,bottom=2cm,left=2cm,right=2cm}

%\flushbottom
%\ruggedbottom

\binoppenalty=5000
\parindent=0pt

\newcommand{\EDS}{\ensuremath{\mathscr{E}}}
\newcommand*{\hm}[1]{#1\nobreak\discretionary{}%
{\hbox{$\mathsurround=0pt #1$}}{}}
\newcommand{\divisible}{\mathop{\raisebox{-2pt}{\vdots}}}
\RequirePackage{caption2}
\renewcommand\captionlabeldelim{}
\renewcommand{\theequation}{\arabic{equation}}
\def\hm#1{#1\nobreak\discretionary{}{\hbox{$#1$}}{}}
\newcommand{\bbskip}{\bigskip \bigskip}

\newtheorem{Lemma}{Лемма}
\theoremstyle{definiton}
\newtheorem{Remark}{Замечание}
%%\newtheorem{Def}{Определение}
\newtheorem{Claim}{Утверждение}
\newtheorem{Cor}{Следствие}
\newtheorem{Theorem}{Теорема}
\newtheorem{Th}{Теорема}
\newtheorem{Example}{Пример}
\newtheorem*{known}{Теорема}
\def\proofname{Доказательство}
\theoremstyle{definition}
\newtheorem{Def}{Определение}
\newtheorem{De}{Определение}
%% \newenvironment{Example} % имя окружения
%% {\par\noindent{\bf Пример.}} % команды для \begin
%% {\hfill$\scriptstyle\qed$} % команды для \end



%%\DeclareMathOperator{\tg}{tg}
%%\DeclareMathOperator{\ctg}{ctg}

\let\leq\leqslant
\let\geq\geqslant



% Remove brackets from numbering in List of References
\makeatletter
\renewcommand{\@biblabel}[1]{\quad#1.}
\makeatother
% Header and Footer with logo
\usepackage{lastpage,fancyhdr,graphicx}
\usepackage{epstopdf}
\pagestyle{myheadings}
\pagestyle{fancy}
\fancyhf{}
\fancyfootoffset{0in}

%\fancyhead[LE, RO]{Летняя Школа <<Phystech.International>>}
%\fancyhead[LO, RE]{\thepage}

\newpagestyle{main}{%
	\setheadrule{.4pt}%
	\sethead
		[\subsectiontitle][][\thepage]
		{\thepage}{}{\sectiontitle} }
\pagestyle{main}
\renewcommand{\headrulewidth}{0pt}  % убрать разделительную линию


%\renewcommand{\footrule}{\hrule height 0.5pt \vspace{2mm}}


\begin{document}

%\begin{titlepage}
%\wpYoffset=0pt
%\ThisCenterWallPaper{0.65}{back}

%\begin{center}


% Upper part of the page
%\includegraphics[width=0.15\textwidth]{./logo}\\[1cm]
%\vspace*{-40pt}
%\textsc{\large Московский Физико-Технический Институт\\ (государственный университет)}\\
%\vspace{12pt}
%\textsc{\large Межвузовский центр воспитания и развития талантливой молодежи в области естественно-математических наук <<Физтех-центр>>}\\
%\vspace{300pt}

% Title
%\HRule \\[0.4cm]
%{ \LARGE \bfseries Вычисление сумм}\\


%\vfill


% Bottom of the page
%\raggedleft{Составитель: Останин П.~А.}
%\vspace{40pt}
%\begin{center}{\large 2016 г. }\end{center}
%22 июня 2011 г.


%\end{center}

%\end{titlepage}

\newpage

\setcounter{page}{2}


%\tableofcontents

\newpage


\section*{Вывод основного уравнения}
\sectionmark{Вывод основного уравнения}

\subsection*{Уравнение непрерывности}
\subsectionmark{Уравнение непрерывности}

Ионосфера~---~это ионизованная часть верхней атмосферы, приблизительно от 60~км до 1000~км, целиком окружающая Землю. Основной источник плазмы~---~фотоионизация нейтральных молекул под действием солнечного ультрафиолета и рентгеновского излучения. Ионы вступают в химические реакции с нейтральными молекулами, рекомбинируют с электронами и диффундируют в другие высоты или перемещаются нейтральным ветром. Но диффузия и перенос подвержены влиянию собственного магнитного поля Земли.

В средних широтах плазма слабо подвержена влиянию электромагнитного поля Земли, но может двигаться вдоль $\vec{B}$.

\bigskip

В рассматриваемом приближении моделируется эволюция концентрации $n_e$ электронов во времени и пространстве в верхней ионосфере (в F-слое).

\medskip

Основным используемым уравнением по существу является уравнение неразрывности для электронной концентрации, выражающее закон сохранения массы: $\dfrac{\partial n_e}{\partial t}+\vec{\nabla}(n_e \vec{u})=P_1-kn_e$, где $u$~---~средняя скорость диффузии. Слагаемые в правой части выражают наличие процессов образования (прямой ионизации при столкновении $O$ и $O^+$) и потерь (процессов рекомбинации).

Считаем $n_e=n_i = n(O^+)$, т.~е. рассматриваем только электроны и ионы кислорода (в F-слое больше всего ионов $O^+$), а плазму считаем квазинейтральной (в связи с этим уравнение непрерывности можно было записать в том же виде и для $n_i$ вместо $n_e$).

\bigskip

За перенос в верхней атмосфере в рассматриваемой модели отвечает амбиполярная диффузия. Её суть заключается в следующем: масса электрона гораздо меньше, чем масса иона кислорода, вследствие чего электроны и ионы разделяются по высоте. Вследствие этого разделяются и заряды, и возникает электростатическое поле. Это поле препятствует дальнейшему разделению слоёв. После его создания электроны и ионы движутся как единый газ.

Для вывода уравнения амбиполярной диффузии используем гораздо более общее уравнение движения: $$nm\dfrac{D\vec{u}}{Dt}+\vec{\nabla} p + \vec{\nabla}\cdot \hat{\tau} - ne(\vec{E}+\vec{u}\times \vec{B})+nm[-\vec{G}+2\vec{\Omega}\times \vec{u}+\vec{\Omega}\times(\vec{\Omega}\times\vec{r})]=$$
$$=\sum_t nm\nu_t (\vec{u}_t-\vec{u})+ \vec{f}(\vec{q})$$
В этом уравнении $\dfrac{D}{Dt}=\dfrac{\partial}{\partial t}+(\vec{u}; \vec{\nabla})\vec{u}$~---~полная производная; $u$~---~скорость диффузии; $n$~---~концентрация электронов; $m$~---~масса электрона; $\hat{\tau}$~---~тензор напряжений; $\vec{G}$~---~ускорение свободного падения; в квадратных скобках помимо $\vec{G}$ стоят слагаемые, связанные с неинерциальностью системы отсчета, вызыванной вращением Земли с угловой скоростью $\vec{\Omega}$; в правой части сумма отвечает за столкновения электронов с различными частицами (разные типы частиц нумеруются индексом $t$), а последнее слагаемое зависит от тепловых потоков, которые в частично ионизованной плазме малы и отбрасываются. Уравнение записано в системе координат, связанной с Землёй.

\bigskip

Далее используем т.~н. диффузионную аппроксимацию: отбросим всю полную производную в силу её малости по сравнению с градиентом давления:

\begin{itemize}

\item[•] $\dfrac{|nm(\vec{u}; \vec{\nabla})\vec{u}|}{|\vec{\nabla}p|}\approx \dfrac{nmu^2/L}{nkT}\approx \dfrac{u^2}{kT/m}=M^2$~---~число Маха. При малых числах Маха (т.~е. при дозвуковых течениях) первое слагаемое можно отбросить.

\item[•] $\dfrac{|nm\partial\vec{u}/\partial t|)\vec{u}}{|\vec{\nabla}p|}\approx \dfrac{L}{\tau} \dfrac{u}{kT/m}\approx M\dfrac{L/\tau}{\sqrt{kT/m}}$, где $\tau$ и $L$~---~характерное время процессов в плазме и характерный размер области, в которой находится рассматриваемая часть плазмы.

\end{itemize}

Для медленно меняющихся и дозвуковых потоков диффузионная аппроксимация применима.

\bigskip

Помимо этого предполагаем квазинейтральность: $n_e=n_i$, движение электронов с ионами единым целым: $n_e \vec{u}_e=n_i \vec{u}_i$. 

Как уже упоминалось выше, считаем, что имеются ионы всего одного вида~---~кислорода. Это справедливо в рассматриваемом F-слое (выше $\approx 130$~км и до $1000$~км).

Запишем в приведенных приближениях общее уравнение движения для электронов и ионов в проекциях на магнитные силовые линии (индекс $\parallel$ указывает соответствующую проекцию):

$$\begin{cases}
\vec{\nabla}_\parallel p_i + (\vec{\nabla}\cdot \hat{\tau}_i)_\parallel + n_ie\vec{E}_\parallel-n_im_i\vec{G}_\parallel=\\
\textrm{ }\textrm{ }\textrm{ }\textrm{ }\textrm{ }=n_im_i\nu_{ie}(\vec{u}_e-\vec{u}_i)_\parallel+n_im_i\nu_{in}(\vec{u}_n-\vec{u}_i)_\parallel\\
\vec{\nabla}_\parallel p_e + (\vec{\nabla}\cdot \hat{\tau}_e)_\parallel - n_ee\vec{E}_\parallel-n_em_e\vec{G}_\parallel=\\
\textrm{ }\textrm{ }\textrm{ }\textrm{ }\textrm{ }=n_em_e\nu_{ei}(\vec{u}_i-\vec{u}_e)_\parallel+n_em_e\nu_{en}(\vec{u}_n-\vec{u}_e)_\parallel
\end{cases}$$

Теперь исключим из уравнений магнитное поле. Сложив эти уравнения и учтя, что $n_e=n_i, \vec{u}_e=\vec{u}_i, n_im_i\nu_{ei}=n_em_e\nu_{ei}$, получим уравнение, не содержащее электрического поля, возникшего вследствие разделения электронов и ионов по слоям:
$$\vec{\nabla}_\parallel (p_i+p_e) + (\vec{\nabla}\cdot (\hat{\tau}_i+\hat{\tau}_e))_\parallel-n_i(m_i+m_e)\vec{G}_\parallel=n_i(m_i\nu_{in}+m_e\nu_{en})(\vec{u}_n-\vec{u}_i)_\parallel$$

Теперь можно отбросить все слагаемые, содержащие массу электрона по сравнению с такими же слагаемыми, но уже с массой иона. После этого заменим давление $p_i=n_ikT_i, p_e=n_ikT_e$ и обозначим $T_p=\dfrac{1}{2}(T_e+T_i)$. Выразив из правой части векторную разность скоростей, получим закон амбиполярной диффузии, дающий характеристику средней скорости диффундирующих ионов: 
$$\vec{u}_{i\parallel} = \vec{u}_{n\parallel} - \dfrac{2kT_p}{m_i\nu_{in}}\left(\dfrac{1}{n_i}\vec{\nabla}_\parallel n_i+\dfrac{1}{T_p}\vec{\nabla}_\parallel T_p-\dfrac{m_i\vec{G}_\parallel}{2kT_p}+\dfrac{(\vec{\nabla}\cdot\hat{\tau_i})_\parallel}{2n_ikT_p}\right)$$

Обозначим $D_a=\dfrac{2kT_p}{m_i\nu_in}$~---~коэффициент амбиполярной диффузии.

\subsection*{Окончательное получение уравнения}
\subsectionmark{Окончательное получение уравнения}

Заметим, что вывод уравнения амбиполярной диффузии полностью сохраняется, если считать, что диффузия происходит вдоль вертикальной оси $z$ и проекции брать на эту ось. Действительно, векторное произведение $\vec{B}\times\vec{u}$ в проекции также даст ноль, как и в случае проецирования на $\vec{B}$.

После получения уравнения амбиполярной диффузии, можно подставить найденную скорость в дивергентный член уравнения непрерывности. Отбросим также тензор напряжений, оставив только давление. После этого задача сведется к одномерному уравнению:
$$\dfrac{\partial n_e}{\partial t}=P_1-kn_e + \dfrac{\partial}{\partial z}\left(D_a\left[\dfrac{\partial n_e}{\partial z}+n_e\left(\dfrac{1}{T_p}\dfrac{\partial T_p}{\partial z}+\dfrac{m_ig}{2kT_p}\right)\right]\right).$$

\newpage

\section*{Численное моделирование}
\sectionmark{Численное моделирование}

\subsection*{Разностная схема}
\subsectionmark{Разностная схема}

Рассмотрим дифференциальное уравнение $$\dfrac{\partial n_e}{\partial t}=P_1-kn_e + \dfrac{\partial}{\partial z}\left(D_a\left[\dfrac{\partial n_e}{\partial z}+n_e\left(\dfrac{1}{T_p}\dfrac{\partial T_p}{\partial z}+\dfrac{m_ig}{2kT_p}\right)\right]\right).$$

В используемой модели температура, концентрация нейтралов, зависимости $D_a(z), k(z)$ и $P_1(z)$~---~внешние параметры. Оценим характерные величины на разных высотах:

\smallskip

\begin{tabular}{|c|c|c|c|}
\hline
&$z_1=200$~км&$z_2=300$~км&$z_3=500$~км\\
\hline
$D_a$, см$^{2}\cdot$с$^{-1}$&$3{,}1\cdot 10^9$&$3{,}4\cdot 10^{10}$&$4{,}2\cdot 10^{12}$\\
\hline
$k$, с$^{-1}$&$5{,}2\cdot 10^{-3}$&$5{,}5\cdot 10^{-5}$&$1{,}3\cdot 10^{-8}$\\
\hline
$P_1$, см$^{-3}\cdot$с$^{-1}$&$1{,}5\cdot 10^3$&$1{,}2\cdot 10^{2}$&$1{,}3$\\
\hline
$u_\textrm{эфф}/D_a$, см$^{-1}$&$4{,}8\cdot 10^{-8}$&$4{,}5\cdot 10^{-8}$&$3{,}6\cdot 10^{-8}$\\
\hline
\end{tabular}

\medskip

Характерное время процессов варьируется от секунд (на нижних высотах) до суток (на верхних). Это означает, что исследуемая система жесткая, используем неявную схему.

\medskip

Для удобства (в силу линейности) разобъем задачу на три.

\begin{itemize}

\item[•] $\dfrac{\partial n_e}{\partial t}=P_1-kn_e.$

Схема: $\dfrac{n_i^{j+1}-n_i^j}{\tau}=p_i^j-kn_i^j+1\Leftrightarrow n_i^{j+1} = \dfrac{n_i^j+p_i^j\tau}{1+k\tau}.$

\bigskip

\item[•] $\dfrac{\partial n_e}{\partial t}=\dfrac{\partial}{\partial z}\left(D_a\dfrac{\partial n_e}{\partial z}\right).$

Схема: $\dfrac{n_i^{j+1}-n_i^j}{\tau} = \dfrac{D_{i+1/2}^j(n_{i+1}^{j+1}-n_i^{j+1})-D_{i-1/2}^j(n_{i}^{j+1}-n_{i-1}^{j+1})}{h^2}$. 

Обозначим для краткости правую часть $\dfrac{1}{h^2}(S\vec{n}^{j+1})_i$.

\bigskip

\item[•] $\dfrac{\partial n_e}{\partial t}=\dfrac{\partial}{\partial z}\left(D_a\left[n_e\left(\dfrac{1}{T_p}\dfrac{\partial T_p}{\partial z}+\dfrac{m_ig}{2kT_p}\right)\right]\right)=\dfrac{\partial}{\partial z}(u_\textrm{эфф.}n_e)$.

Схема: $\dfrac{n_i^{j+1}-n_i^j}{\tau} = \dfrac{u_{i+1}n_{i+1}^{j+1}-u_i n_i^{j+1}}{h}$.

Обозначим для краткости правую часть $\dfrac{1}{h}(A\vec{n}^{j+1})_i$.

\end{itemize}

\subsection*{Монотонность, сохранение массы}
\subsectionmark{Монотонность, сохранение массы}

В силу неотрицательности $n_e$ от схемы требуется свойство монотонности: если на каком-либо шаге $\vec{n}^j \geq 0$, то и на следующем шаге $\vec{n}^{j+1}\geq 0$. Для монотонности необходимо и достаточно неотрицательности элементов матрицы $B$ системы $\vec{n}^{j+1}=B\vec{n}^j$, эквивалентной нашей разностной схеме.

\smallskip

Докажем монотонность для нашей схемы. Перепишем её в виде $$\dfrac{\vec{n}^{j+1}-\vec{n}^j}{\tau}= \vec{p} - k\vec{n}^{j+1}+\dfrac{1}{h^2}S\vec{n}^{j+1}+\dfrac{1}{h}A\vec{n}^{j+1}$$

Здесь $E$~---~единичная матрица подходящей размерности.

Выпишем введенные матрицы $S$ и $A$:

$$S = \left(\begin{smallmatrix}
(-D_{1+1/2}-D_{1-1/2}) & D_{1+1/2}             & 0        & 0  & \cdots & 0 \\
D_{2-1/2}              & (-D_{2+1/2}-D_{2-1/2}) & D_{2+1/2} & 0  & \cdots & 0 \\         
\ddots & \ddots & \ddots & \ddots \\
\cdots & D_{i-1/2} & (-D_{i+1/2}-D_{i-1/2}) & D_{i+1/2} &  \cdots &  \\
\ddots & \ddots & \ddots & \ddots
\end{smallmatrix}\right).$$

$$A=\left(\begin{matrix}
-u_1 &  u_2 & 0   & 0  & \cdots & 0 \\
0    & -u_2 & u_3 & 0  & \cdots & 0 \\         
\ddots & \ddots & \ddots & \ddots \\
\cdots & \cdots & u_i & u_{i+1} &  \cdots &  \\
& & \ddots & \ddots & \ddots\\
& & & & \ddots& u_N\\
& & & & & -u_N
\end{matrix}\right).$$

\smallskip

Первая и последняя строки зависят от краевых условий.

Перепишем систему в виде $$\vec{n}^{j+1}=(E+\tau k E-\dfrac{\tau}{h^2} S-\dfrac{\tau}{h}A)^{-1}(\tau \vec{p}+\vec{n}^j)=$$ $$=\left(E-\dfrac{\tau}{h^2(1+\tau k)}S-\dfrac{\tau}{h(1+\tau k)}A\right)^{-1}\cdot \dfrac{\tau \vec{p}+\vec{n}^j}{\tau k + 1}.$$

Исследуем матрицу $B=\left(E-\dfrac{\tau}{h^2(1+\tau k)}S-\dfrac{\tau}{h(1+\tau k)}A\right)^{-1}$. 

\smallskip
Покажем, что элементы обратной матрицы неотрицательны при $v_k\geq 0$ для всех $k=1\cdots N$. Для этого обратимся к алгоритму Гаусса-Жордана поиска обратной матрицы: проводятся элементарные преобразования строк с приведением рассматриваемой матрицы к единичной. При этом справа записывается единичная матрица подходящего размера, и с ней проводятся все те же преобразования.

\smallskip
Воспользуемся тем фактом, что в рассматриваемой трёхдиагональной матрице $E-\dfrac{\tau}{h^2(1+\tau k)}S-\dfrac{\tau}{h(1+\tau k)}A$ на диагонали стоят положительные элементы, а вне неё~---~отрицательные. В этом случае можно элементарными преобразованиями привести её к единичной в следующем порядке:

\begin{itemize}
\item[1.] Сначала первая строка, домноженная на подходящее неотрицательное число, добавляется ко второй, при этом во второй строке первый элемент обнуляется. Затем вторая строка домножается на неотрицательное число и добавляется к третьей так, чтобы обнулить второй элемент третьей строки. Таким образом в рассматриваемой матрице все элементы ниже главной диагонали обнуляются, а параллельные преобразования с единичной матрицей оставляют элементы неотрицательными.

\item[2.] Далее можно, начиная с предпоследней строки, вычитать из нее предыдущую, домноженную на некоторое положительное число. Таким образом мы обнулим и все элементы выше главной диагонали. Останутся только неотрицательные диагональные элементы, после деления на них (также не меняющего знака элементов в матрице справа) приходим к единичной матрице слева.
\end{itemize}

Получаем неотрицательность всех элементов обратной матрицы, которая и даёт монотонность схемы.
\bigskip

Исходное уравнение (в случае отсутствия рождения $P_1$ и уничтожения $(-kn_e)$) имеет закон сохранения массы. В непрерывном случае проинтегрируем уравнение на всём исследуемом отрезке высот $[H_1; H_2]$: 

$$\int_{H_1}^{H_2} \dfrac{\partial n}{\partial t} dz = D(H_2)\dfrac{\partial n}{\partial z}\bigg|_{H_2}-D(H_1)\dfrac{\partial n}{\partial z}\bigg|_{H_1}+u(H_2)n(H_2)-u(H_1)n(H_1).$$

При должном выборе граничных условий можно занулить слагаемые в правой части и получить $\displaystyle\int_{H_1}^{H_2}\dfrac{\partial n}{\partial t} dz = 0$.

$\bigg($Если учесть рождение и уничтожение, то получим $\displaystyle\int_{H_1}^{H_2}\dfrac{\partial n}{\partial t} dz \hm= \displaystyle\int_{H_1}^{H_2} (P_1-kn)dz\bigg)$.

\bigskip

В дискретном случае также справедлив закон сохранения массы. Дискретизируем производные по $z$, сохранив сначала непрерывность по $t$: $$\dfrac{\partial}{\partial t} n_i = (D_{i+1/2}(n_{i+1}-n_i)-D_{i-1/2}(n_i-n_{i-1}))\dfrac{1}{h^2}+(u_{i+1}n_{i+1}-u_i n_i)\dfrac{1}{h}.$$

Суммируя по всем узлам $i$, получим: $$\dfrac{\partial}{\partial t} \displaystyle\sum_{i=1}^{N} n_i = (D_{N-1/2}(n_{N}-n_{N-1})-D_{1/2}(n_1-n_0))\dfrac{1}{h^2}+(u_{N}n_{N}-u_1 n_1)\dfrac{1}{h}.$$

Как и в непрерывном случае, считаем, что краевые условия обнуляют правую часть. Тогда $\dfrac{\partial}{\partial t} \displaystyle\sum_{i=1}^{N} n_i =0$.

Теперь дискретизируем и оператор производной по времени: $$\displaystyle\sum_{i=1}^{N} \dfrac{n_i^{j+1}-n_i^j}{\tau}=0.$$ Домножая на $\tau$, получим $\displaystyle\sum_{i=1}^{N} n_i^{j+1}=\displaystyle\sum_{i=1}^{N} n_i^j$, что и требовалось.

%\newpage
%\begin{thebibliography}{00} % библиография
%\bibitem{}
%Виленкин~Н.~Я., Виленкин~А.~Н., Виленкин~П.~А. Комбинаторика.~---
%М.: ФИМА, МЦНМО, 2010.~---~400~с.
%\end{thebibliography}
\end{document}





