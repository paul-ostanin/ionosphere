\documentclass[2pt, a4paper, fleqn]{extarticle}

\usepackage[T2A]{fontenc}
\usepackage[utf8]{inputenc}
\usepackage[english,russian]{babel}
\usepackage{url}
%\usepackage{pscyr}

%\renewcommand{\rmdefault}{ftm}
\usepackage{setspace}
\onehalfspacing

\usepackage{changepage}
\usepackage{indentfirst} %первый абзац
%%\usepackage{moreverb}
\usepackage[noend]{algorithmic}
\usepackage{amssymb, amsmath, multicol,amsthm}
%%
\usepackage{enumitem, multicol}
\usepackage{titleps,lipsum}
%%
\usepackage{mathrsfs}
\usepackage{verbatim}
\usepackage{pb-diagram}
\usepackage{graphicx}
\graphicspath{ {images/} }
\usepackage{wrapfig}
\usepackage{xcolor}
\definecolor{new}{RGB}{255,184,92}
\definecolor{news}{RGB}{112,112,112}
\usepackage{wallpaper}
\usepackage{float}
\usepackage{hyperref}
\hypersetup{
%colorlinks=true,%
%linkcolor=news,%
linkbordercolor=new,
}



\usepackage{geometry}
\geometry{top=1cm,bottom=2cm,left=1cm,right=1cm}

%\flushbottom
%\ruggedbottom

\binoppenalty=5000
\parindent=0pt

\newcommand{\EDS}{\ensuremath{\mathscr{E}}}
\newcommand*{\hm}[1]{#1\nobreak\discretionary{}%
{\hbox{$\mathsurround=0pt #1$}}{}}
\newcommand{\divisible}{\mathop{\raisebox{-2pt}{\vdots}}}
\renewcommand{\theequation}{\arabic{equation}}
\def\hm#1{#1\nobreak\discretionary{}{\hbox{$#1$}}{}}
\newcommand{\bbskip}{\bigskip \bigskip}



%%\DeclareMathOperator{\tg}{tg}
%%\DeclareMathOperator{\ctg}{ctg}

\let\leq\leqslant
\let\geq\geqslant



% Remove brackets from numbering in List of References
\makeatletter
\renewcommand{\@biblabel}[1]{\quad#1.}
\makeatother

\begin{document}

На первом шаге расщепления решается уравнение $$\dfrac{\partial n_i}{\partial t} = \dfrac{\partial}{\partial z}\bigg[D\sin^2 I \bigg(\dfrac{\partial n_i}{\partial z}+\left(\dfrac{1}{T_p}\dfrac{\partial T_p}{\partial z}+\dfrac{1}{H}\right)n_i\bigg)-$$ $$-\dfrac{1}{a}D\sin I\cos I\left(\dfrac{\partial n_i}{\partial\varphi}+\dfrac{1}{T_p}\dfrac{\partial T_p}{\partial \varphi}n_i\right)\bigg]+[P-kn_i]$$

Рассмотрим следующие варианты разностных схем для этого уравнения. Во всех случаях используем для диффузионного слагаемого схему, получаемую применением формулы центральной разности дважды, а для потокового слагаемого $\dfrac{\partial}{\partial z} (un)$ центральную разность. Различия рассматриваемых схем~---~в аппроксимации слагаемого со смешанной производной. В схемах $1, 3, 4$ вычисление $u_\varphi$ ведётся по формуле с логарифмом $$u_\varphi = \dfrac{\partial \ln n}{\partial\varphi}\approx \dfrac{\ln\frac{n_{i, j+1}}{n_{i, j-1}}}{2\Delta\varphi}.$$ В схеме $2$ для сравнения эта аппроксимация заменена на $$u_\varphi \approx \dfrac{2}{n_{i, j+1}+n_{i, j-1}}\dfrac{n_{i, j+1}+n_{i, j-1}}{2\Delta\varphi}.$$ Отметим также следующее: эксперимент показал что в процессе вычисления величина $n$ никогда не достигает в точности нулевого значения, поэтому вопрос об асимптотической формуле для $u_\varphi$ в нулях $n$ не ставится.

\medskip

{\bf 1.} Схема с центральными разностями:используем для смешанной производной центральную разность в уравнении и в граничном условии. В результате получим стационарное решение, в котором, тем не менее, имеются сильные осцилляции вдоль изменения широты. Период осцилляции соответствует мелкости шага сетки (при дроблении сетки частота осцилляции увеличивается).

Схема: $$\left(\dfrac{\partial n}{\partial t}\right)_i = D_{i+1/2}\dfrac{n_{i+1}-n_i}{h} - D_{i-1/2}\dfrac{n_{i}-n_{i-1}}{h}+\dfrac{(u_{i+1}+u_{\varphi(i+1)})n_{i+1}-(u_{i-1}+u_{\varphi(i-1)})n_{i-1}}{2h}$$

Верхнее граничное условие: $$D_{N+1/2}\dfrac{n_{N+1}-n_{N}}{h}+\dfrac{u_{N+1}n_{N+1}+u_{N}n_{N}}{2h}+\dfrac{u_{\varphi(N+1)}n_{N+1}+u_{\varphi(N)}n_{N}}{2h} = F_{ub} = 0$$

\medskip

{\bf 2.} Единственное отличие этой схемы от схемы $1$~---~в способе аппроксимации и вычисления самой величины $u_\varphi$: используется формула без логарифмов $u_\varphi \approx \dfrac{2}{n_{i, j+1}+n_{i, j-1}}\dfrac{n_{i, j+1}+n_{i, j-1}}{2\Delta\varphi}$. Эксперименты показали, что результаты вычислений слабо зависят от способа этой аппроксимации. Далее везде применяется формула с логарифмом. Расчеты также показали наличие сильных биений в данной схеме.

\medskip

{\bf 3.} Запишем несогласованную схему: в уравнении используем для смешанной производной направленную разность, а в граничном условии~---~соответствующее условие для центральной разности. Это внесёт дополнительную численную диффузию, в результате чего осцилляции прекратятся.

Схема: $$\left(\dfrac{\partial n}{\partial t}\right)_i = D_{i+1/2}\dfrac{n_{i+1}-n_i}{h} - D_{i-1/2}\dfrac{n_{i}-n_{i-1}}{h}+$$ $$+\dfrac{u_{i+1}n_{i+1}-u_{i-1}n_{i-1}}{2h} + $$ $$+\dfrac{|u_\varphi|+u_\varphi}{2}\cdot\dfrac{n_{i+1}-n_i}{h}+\dfrac{|u_\varphi|-u_\varphi}{2}\cdot\dfrac{n_i-n_{i-1}}{h}$$

Верхнее граничное условие: $$D_{N+1/2}\dfrac{n_{N+1}-n_{N}}{h}+\dfrac{u_{N+1}n_{N+1}+u_{N}n_{N}}{2h}+\dfrac{u_{\varphi(N+1)}n_{N+1}+u_{\varphi(N)}n_{N}}{2h} = F_{ub} = 0$$

\medskip

{\bf 4.} Наконец, используем формулу центральной разности и в уравнении, и в граничном условии. Для выяснения корректной согласованной (консервативной) аппроксимации граничного условия исследуем функцию потока в схеме: $$\left(\dfrac{\partial n}{\partial t}\right)_i = D_{i+1/2}\dfrac{n_{i+1}-n_i}{h} - D_{i-1/2}\dfrac{n_{i}-n_{i-1}}{h}+$$ $$+\dfrac{u_{i+1}n_{i+1}-u_{i-1}n_{i-1}}{2h} + $$ $$+\dfrac{|u_\varphi|+u_\varphi}{2}\cdot\dfrac{n_{i+1}-n_i}{h}+\dfrac{|u_\varphi|-u_\varphi}{2}\cdot\dfrac{n_i-n_{i-1}}{h}$$

Перепишем в консервативной форме, явно выделив функцию потока:

$$\left(\dfrac{\partial n}{\partial t}\right)_i = q_{i+1/2}-q_{i-1/2} \Leftrightarrow $$ $$\Leftrightarrow \left(\dfrac{\partial n}{\partial t}\right)_i = \left(D_{i+1/2}\dfrac{n_{i+1}-n_i}{h} + \dfrac{u_{i+1}n_{i+1}+u_in_i}{2h}+\dfrac{|u_\varphi|}{2}\dfrac{n_{i+1}+n_{i}}{h}+\dfrac{u_\varphi}{2}\dfrac{n_{i+1}-n_i}{h}\right) - $$ $$ - \left(D_{i-1/2}\dfrac{n_{i}-n_{i-1}}{h} + \dfrac{u_{i}n_{i}+u_{i-1}n_{i-1}}{2h}+\dfrac{|u_\varphi|}{2}\dfrac{n_{i}+n_{i-1}}{h}+\dfrac{u_\varphi}{2}\dfrac{n_{i}-n_{i-1}}{h}\right)$$

Теперь аппроксимируем граничное условие следующим образом: $$D_{N+1/2}\dfrac{n_{N+1}-n_N}{h}+\dfrac{u_{N+1}n_{N+1}+u_Nn_N}{2h}+\dfrac{|u_\varphi|}{2}\dfrac{n_{N+1}+n_{N}}{h}+\dfrac{u_\varphi}{2}\dfrac{n_{N+1}-n_N}{h} = F_{ub} = 0$$



\end{document}



