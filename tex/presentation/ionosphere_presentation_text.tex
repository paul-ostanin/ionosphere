\documentclass[2pt, a4paper, fleqn]{extarticle}

\usepackage[T2A]{fontenc}
\usepackage[utf8]{inputenc}
\usepackage[english,russian]{babel}
\usepackage{url}
%\usepackage{pscyr}

%\renewcommand{\rmdefault}{ftm}
\usepackage{setspace}
\onehalfspacing

\usepackage{changepage}
\usepackage{indentfirst} %первый абзац
%%\usepackage{moreverb}
\usepackage[noend]{algorithmic}
\usepackage{amssymb, amsmath, multicol,amsthm}
%%
\usepackage{enumitem, multicol}
\usepackage{titleps,lipsum}
%%
\usepackage{mathrsfs}
\usepackage{verbatim}
\usepackage{pb-diagram}
\usepackage{graphicx}
\graphicspath{ {images/} }
\usepackage{wrapfig}
\usepackage{xcolor}
\definecolor{new}{RGB}{255,184,92}
\definecolor{news}{RGB}{112,112,112}
\usepackage{wallpaper}
\usepackage{float}
\usepackage{hyperref}
\hypersetup{
%colorlinks=true,%
%linkcolor=news,%
linkbordercolor=new,
}



\usepackage{geometry}
\geometry{top=1cm,bottom=2cm,left=1cm,right=1cm}

%\flushbottom
%\ruggedbottom

\binoppenalty=5000
\parindent=0pt

\newcommand{\EDS}{\ensuremath{\mathscr{E}}}
\newcommand*{\hm}[1]{#1\nobreak\discretionary{}%
{\hbox{$\mathsurround=0pt #1$}}{}}
\newcommand{\divisible}{\mathop{\raisebox{-2pt}{\vdots}}}
\renewcommand{\theequation}{\arabic{equation}}
\def\hm#1{#1\nobreak\discretionary{}{\hbox{$#1$}}{}}
\newcommand{\bbskip}{\bigskip \bigskip}



%%\DeclareMathOperator{\tg}{tg}
%%\DeclareMathOperator{\ctg}{ctg}

\let\leq\leqslant
\let\geq\geqslant



% Remove brackets from numbering in List of References
\makeatletter
\renewcommand{\@biblabel}[1]{\quad#1.}
\makeatother

\begin{document}

В работе рассматривается решение задачи по построению динамической модели Земной ионосферы (для 100-500 км) При разработке первой версии модели ионосферы используются приближения: рассмотрение только F слоя, динамическое преобладание амбиполярной диффузии, одноионная постановка, дипольное магнитное поле Земли, приближение совпадения географических и магнитных полюсов, квазинейтральность плазмы.

В модели решается уравнение неразрывности для концентрации электронов и положительных ионов атомарного кислорода с учётом фотохимических процессов (ионизации и рекомбинации) и амбиполярной диффузии вдоль силовых линий Земного магнитного поля. Решаемое уравнение сводится к уравнению переноса и диффузии, с существенной ролью динамических процессов в верхних слоях и плазмохимических процессов в нижних. С учётом замечаний выше уравнение неразрывности принимает следующий вид:
$$\dfrac{\partial n_i}{\partial t} = -div(n_i \vec{u}_\parallel)-div\left(n_i\dfrac{1}{B^2}[\vec{E}\times \vec{B}] \right)+$$ $$+div\left(D\left[\nabla_\parallel n_i +n_i\dfrac{1}{T_p}\nabla_\parallel T_p - \dfrac{n_i m_i}{2kT_p}\vec{g}_\parallel\right]\right)+[P-k_in_i]$$ В этом уравнении $D$~---~коэффициент амбиполярной диффузии, $\vec{u}$~---~нейтральный ветер, $T_p$~---~полусумма электронной и ионной температур.

\medskip

{\bf Слайд 2: входящие в уравнение параметры}

Входящие в уравнение в качестве внешних параметров функции фотоионизации, рекомбинации, температуры нейтралов, электронов и ионов, а также концентрации молекул $N_2$, $O_2$ и $O$ задаются аналитическими формулами. 

Температуры вычисляем по аналитическим формулам, представленным на слайде 

($R=\dfrac{R_0}{M_{air}}\approx 287$~Дж$\cdot$кг$^{-1}\cdot$K$^{-1}$, $R_0$~---~универсальная газовая постоянная).

Для концентраций используем Больцмановское распределение по высоте.

Функции рекомбинации и фотоионизации (в дневное время) можно приближенно вычислять по следующим формулам.

\medskip

{\bf Слайд 3: метод расщепления, три постановки}

Выберем последовательно в полученном трёхмерном уравнении ключевые процессы, формирующие поле скоростей. Затем реализуем модель поэтапно, учитывая каждый раз новые поправки и сравнивая новое решение с предыдущим.

В первом приближении оставим только фотохимию и амбиполярную диффузию, считая, что диффузия происходит только вдоль оси $z$. При этом получим следующее одномерное уравнение для электронной концентрации $n$.

В уравнении $D$~---~коэффициент амбиполярной диффузии, $u = D\left(\dfrac{1}{T_p}\dfrac{\partial T_p}{\partial z}+\dfrac{m_ig}{2kT_p}\right)$~---~эффективная скорость, $P$ и $kn$~---~слагаемые, отвечающие процессам ионизации при столкновении $O$ и $O+$ и рекомбинации соответственно.

Следующим шагом учтём широтную зависимость в уравнении. Простейший способ~---~замена коэффициента диффузии $D$ на $D\sin^2I$, где $I$~---~угол наклонения магнитных силовых линий, $I \approx \arctg(2\tg \varphi)$, $\varphi$~---~широта ($\varphi \in [-90^\circ; +90^\circ]$).

Особенность данной постановки состоит в том, что на экваторе при $\varphi=0$ уравнение вырождается. Этот эффект не соответствует никакому физическому явлению, уравнение не описывает физические процессы на экваторе.

Более точный учёт широтной зависимости решения приводит к двумерной задаче, включающей диффузию вдоль оси $z$ в проекции (со смешанной производной).

\medskip

{\bf Слайд 5: требования к схемам и свойства решения}

Уравнения, описанные в предыдущем разделе, имеют ряд особенностей, которые необходимо учитывать при численном моделировании.

От разностной схемы требуется сохранение неотрицательности значений $n$ на следующем временном слое, если это свойство было выполнено на предыдущем (монотонность схемы), а также выполнение закона сохранения массы, имеющего места в дифференциальной задаче. Эти требования связаны с отсутствием физического смысла у решений, не сохраняющих массу или содержащих отрицательные значения концентрации.

Кроме этого рассматриваемое уравнение при неизменных $P$ и $k$ имеет стационарное решение.

Вблизи нижней границы влияние диффузионного слагаемого и переноса пренебрежимо малы по сравнению с процессами фотохимии. Напротив, на верхней части исследуемого высотного интервала преобладают диффузионные процессы, а $P$ и $k$ уже не играют роли. Важной особенностью рассматриваемой задачи является изменение входящих в уравнение коэффициентов $D, P, k, u$ на рассматриваемом отрезке на несколько порядков. 

\newpage

{\bf Слайд 6: используемые схемы}

Перейдем к получению используемых разностных схем. Введём следующие обозначения для шагов по пространству: $h_i = z_{i+1}-z_i, h_{i+1/2}=z_{i+1/2}-z_{i-1/2}$.
В точке $z=z_i$ для слагаемого $\dfrac{\partial}{\partial z}D\dfrac{\partial n}{\partial z}$ в разностных схемах используется следующая аппроксимация, полученная двойным применением формулы центральной разности на отрезках $[z_{i-1};z_i]$ и $[z_i; z_{i+1}]$

Соответственно, в численных экспериментах протестированы три различные разностные схемы: 
\begin{itemize}
\item[•] В схеме $1$ потоковый член и граничное условие аппроксимируются с помощью направленных разностей; 
\item[•] В схеме $2$ только потоковый член в уравнении записывается с помощью направленных разностей, а граничное условие всё еще использует центральные разности (несогласованное граничное условие);
\item[•] Наконец, схема $3$ имеет согласованные граничное условие и схему, записанные с помощью центральных разностей.
\end{itemize}


\smallskip

Для уравнения со смешанной производной введём обозначение $u_\varphi=-\dfrac{1}{a}D\sin I \cos I\dfrac{\partial \ln n}{\partial \varphi}=-\dfrac{1}{a}D\sin I \cos I\dfrac{1}{n}\dfrac{\partial n}{\partial \varphi}.$ Для рассматриваемого уравнения $u_\varphi$~---~это добавка к эффективной скорости, связанная со смешанной производной по $\varphi$.

\medskip

{\bf Слайды 7-8: Стационарное решение одномерной задачи}

Прежде всего обратимся к одномерной задаче для $z$-диффузии без проекций. Исследуемая задача имеет не зависящее от времени решение, при итерациях по времени происходит устсановление решения во всех трёх указанных схемах. 

Результаты расчетов (стационарные решения в зависимости от разного количества узлов по пространству) представлены на следующих графиках (по горизонтальной оси масштаб выбран логарифмическим). Соответственно, $80$ и $2000$ узлов отвечают шагам по времени $5$~км и $0{,}2$~км.

Видим, что для схемы с несогласованным граничным условием необходим гораздо более мелкий шаг, чтобы сойтись к нужному решению.

\medskip
 
{\bf Слайд 9: чувствительность к изменению внешних параметров}

Полученное решение позволяет исследовать чувствительность к изменению различных входящих в уравнение внешних параметров: температурам, концентрациям нейтральных молекул, фотоионизации и рекомбинации. При исследовании в каждом случае вычислялось стационарное решение при изменённом параметре.

Варьирование входящих в уравнение параметров показывает, что наибольшую чувствительность решение имеет к температуре нейтральных молекул. 

\medskip

{\bf Слайд 10: учёт широтной зависимости}

Учтём теперь широтную зависимость в уравнении.  Для исследования и сравнения качественных отличий полученных результатов установим не зависящую от времени ионизацию $P(z, t) \equiv P_0(z)$ и изучим сходимости к стационарным решениям с одних и тех же начальных условий~---~вектора с компонентами, равными единице (такой вектор отвечает <<почти нулевому>> решению). 

Полученные решения при широтах $\varphi = -30^\circ$ и $\varphi=-2^\circ$ представлены на графиках. Видно, что решения существенно различны вблизи экватора. На самом деле рассматриваемые уравнения не описывают физику вблизи экватора, поэтому решения, получаемые по этим схемам некорректны, необходимо уточнять их, используя полную трёхмерную модель.

\medskip

{\bf Слайд 11: Моделирование суточного хода в одномерной модели}

В ходе численного эксперимента по моделированию суточного хода в одномерной модели вычисляется стационарное решение одномерной задачи при дневном значении $P(z)$, а затем итерации по времени продолжаются с уже меняющимся $P(z, t)$ в соответствии со введённой формулой.

Результаты представлены следующим графиком~---~трёхмерной поверхностью, построенной над плоскостью $(z, t)$.

Видно, что за сутки решение восстанавливается. Кроме того, после зануления $P$ при зенитных углах больше $90^\circ$ начинается спад электронной концентрации, сопровождающийся изломом по времени (в соответствующий  момент $P$, входящее в уравнение, также терпит излом).

Отметим также, что при обнулении $P$ решение (начиная со стационарного) падает почти до нуля приблизительно за $6$ часов.

\newpage

Дополнение (если спросят): формулы для суточного хода.
Для моделирования суточного изменения вертикального профиля добавим зависимость от времени в слагаемое $P$, отвечающее фотоионизации.

Используем формулу $$P(z, t) =\begin{cases}
P_0(z)e^{\tau_0(z)(1-\sec\chi)}, |\chi|\leq\dfrac{\pi}{2}\\
0, |\chi|\geq\dfrac{\pi}{2}
\end{cases}$$

Здесь использованы следующие обозначения: $P_0(z)$~---~фотоионизация в дневное время, $\chi$~---~зенитный угол Солнца (угол между направлением на Солнце и нормалью к земной поверхности), $\tau_0(z)$~---~оптическая толщина, для вычисления которой используется формула $$\tau_0(z)=\sum_{i = N_2, O_2, O} \sigma_i^{abs}\left[\dfrac{R_0T_n}{M_i g}n_i(z)\right]= \dfrac{R_0T_n}{g}\left(\sigma_{N_2}^{abs}\dfrac{n_{N_2}(z)}{M_{N_2}}+\sigma_{O_2}^{abs}\dfrac{n_{O_2}(z)}{M_{O_2}}+\sigma_{O}^{abs}\dfrac{n_{O}(z)}{M_{O}}\right).$$

Константы $\sigma_i^{abs}$ для трёх типов нейтральных молекул известны и равны соответственно $\sigma_{N_2}^{abs}=1{,}5\cdot 10^{-17}$~см$^2$, $\sigma_{O_2}^{abs}=2\cdot 10^{-17}$~см$^2$, $\sigma_{O}^{abs}=1\cdot 10^{-17}$~см$^2$.

Характерные величины оптической толщины на различных высотах представлены в следующей таблице:

\smallskip

$$\begin{tabular}{|c|c|c|c|}
\hline
&$z_1=100$~км&$z_2=300$~км&$z_3=500$~км\\
\hline
$\tau_0$&$4\cdot 10^2$&$3\cdot 10^{-1}$&$2\cdot 10^{-4}$\\
\hline
\end{tabular}$$

\medskip


В предложенной формуле для фотоионизации время в качестве параметра входит лишь в зенитный угол. Кусочное задание функции $P(z, t)$ связано с приближением отсутствия фотоионизации в ночное время (Солнце не заходит за горизонт лишь при зенитных углах, не превосходящих $90^\circ$).

Зависимость зенитного угла от времени даётся следующими формулами: $$\cos\chi = \sin\varphi\cdot\sin\delta-\cos\varphi\cdot\cos\delta\cdot\cos\omega t$$

Здесь $\omega$~---~угловая скорость вращения Земли, $\varphi$~---~широта, а $\delta$~---~склонение Солнца, тангенс которого определяется формулой $$\tg\delta = \tg 23{,}5^\circ \cdot \sin\left(2\pi\cdot\dfrac{d-80}{365}\right),$$ где $d$~---~номер дня от начала года.

\textit{ Дополнение о гран. условиях}

\textit{ Нижнее граничное условие (условие Дирихле) аппроксимируется точно, а на верхней границе условие постоянства потока может быть записано несколькими способами. Для данной одномерной задачи используем две различных аппроксимации этого условия:}


\textit{В первом случае поток $\dfrac{\partial n}{\partial z}+\dfrac{u_N}{D_N}\cdot n_N=F$ аппроксимируется с помощью центральных разностей по пространству, что соответствует схеме $n_N-n_{N-1}+u_N/D_N\cdot h_N\cdot n_N = F\cdot h_N$}

\textit{Во втором случае для схемы центральных разностей запишем согласованную схему для верхнего граничного случая, получаемую с помощью интегрирования уравнения на $N$-ом шаге по пространству между двумя соседними полуцелыми узлами, а также учёта равенства потока на верхнем полуцелом узле заданной величине $F$: $h_{N+1/2}\dfrac{n^{j+1}-n^j}{\tau}= F - D_{N-1/2}\dfrac{n_N-n_{N-1}}{h_{N-1}}-\dfrac{1}{2}(u_{N-1}n_{N-1}^{j+1}+u_{N}n_{N}^{j+1})$
}
 
 Характерные величины на нескольких высотах представлены в следующей таблице: 

\smallskip

\begin{tabular}{|c|c|c|c|}
\hline
&$z_1=200$~км&$z_2=300$~км&$z_3=500$~км\\
\hline
$D$, см$^{2}\cdot$с$^{-1}$&$3{,}1\cdot 10^9$&$3{,}4\cdot 10^{10}$&$4{,}2\cdot 10^{12}$\\
\hline
$k$, с$^{-1}$&$5{,}2\cdot 10^{-3}$&$5{,}5\cdot 10^{-5}$&$1{,}3\cdot 10^{-8}$\\
\hline
$P_1$, см$^{-3}\cdot$с$^{-1}$&$1{,}5\cdot 10^3$&$1{,}2\cdot 10^{2}$&$1{,}3$\\
\hline
$u_\textrm{эфф}/D$, см$^{-1}$&$4{,}8\cdot 10^{-8}$&$4{,}5\cdot 10^{-8}$&$3{,}6\cdot 10^{-8}$\\
\hline
\end{tabular}

\medskip


\end{document}



