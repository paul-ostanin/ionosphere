\documentclass[2pt, a4paper, fleqn]{extarticle}

\usepackage[T2A]{fontenc}
\usepackage[utf8]{inputenc}
\usepackage[english,russian]{babel}
\usepackage{url}
%\usepackage{pscyr}

%\renewcommand{\rmdefault}{ftm}
\usepackage{setspace}
\onehalfspacing

\usepackage{changepage}
\usepackage{indentfirst} %первый абзац
%%\usepackage{moreverb}
\usepackage[noend]{algorithmic}
\usepackage{amssymb, amsmath, multicol,amsthm}
%%
\usepackage{enumitem, multicol}
\usepackage{titleps,lipsum}
%%
\usepackage{mathrsfs}
\usepackage{verbatim}
\usepackage{pb-diagram}
\usepackage{graphicx}
\graphicspath{ {images/} }
\usepackage{wrapfig}
\usepackage{xcolor}
\definecolor{new}{RGB}{255,184,92}
\definecolor{news}{RGB}{112,112,112}
\usepackage{wallpaper}
\usepackage{float}
\usepackage{hyperref}
\hypersetup{
%colorlinks=true,%
%linkcolor=news,%
linkbordercolor=new,
}



\usepackage{geometry}
\geometry{top=1cm,bottom=2cm,left=1cm,right=1cm}

%\flushbottom
%\ruggedbottom

\binoppenalty=5000
\parindent=0pt

\newcommand{\EDS}{\ensuremath{\mathscr{E}}}
\newcommand*{\hm}[1]{#1\nobreak\discretionary{}%
{\hbox{$\mathsurround=0pt #1$}}{}}
\newcommand{\divisible}{\mathop{\raisebox{-2pt}{\vdots}}}
\renewcommand{\theequation}{\arabic{equation}}
\def\hm#1{#1\nobreak\discretionary{}{\hbox{$#1$}}{}}
\newcommand{\bbskip}{\bigskip \bigskip}



%%\DeclareMathOperator{\tg}{tg}
%%\DeclareMathOperator{\ctg}{ctg}

\let\leq\leqslant
\let\geq\geqslant



% Remove brackets from numbering in List of References
\makeatletter
\renewcommand{\@biblabel}[1]{\quad#1.}
\makeatother

\begin{document}

{\bf Слайд 2: Введение}
В работе рассматривается решение задачи по построению динамической модели Земной ионосферы (для 100-500 км) с целью дальнейшего включения этой модели в качестве вычислительного блока в совместную модель верхней атмосферы: работа является частью реализуемого в ИВМ РАН исследования по моделированию глобального состояния верхней атмосферы Земли и направлена на разработку согласованной глобальной модели ионосферы и термосферы высокого уровня. 

Актуальность данной задачи обусловлена повышенным в последние годы практическим интересом к исследованию и прогнозированию космической погоды, что связано с особой ролью состояния ионосферы для систем глобальной радиосвязи, спутниковых систем, а также для космической отрасли в целом. Состояние системы термосфера-ионосфера определяет как характеристики движения низкоорбитальных спутников и космических аппаратов, так и условия для распространения радиосигналов, обеспечивающих бесперебойную работу систем дальней радиосвязи, радиолокации, а также навигационных систем глобального спутникового позиционирования. 

\medskip

{\bf Слайд 3: Постановка задачи.}
При разработке первой версии модели ионосферы используются приближения: рассмотрение только F слоя, динамическое преобладание амбиполярной диффузии, одноионная постановка, дипольное магнитное поле Земли, приближение совпадения географических и магнитных полюсов, квазинейтральность плазмы.

В модели решается уравнение неразрывности для концентрации электронов (или положительных ионов атомарного кислорода в силу квазинейтральности плазмы) с учётом фотохимических процессов (ионизации и рекомбинации) и амбиполярной диффузии вдоль силовых линий Земного магнитного поля. Решаемое уравнение сводится к уравнению переноса и диффузии, с существенной ролью динамических процессов в верхних слоях и плазмохимических процессов в нижних. С учётом замечаний выше уравнение неразрывности имеет вид, представленный на слайде. В этом уравнении $D$~---~коэффициент амбиполярной диффузии, $\vec{u}$~---~нейтральный ветер, $T_p$~---~полусумма электронной и ионной температур.

\medskip

{\bf Слайд 4: входящие в уравнение параметры.}

Входящие в уравнение в качестве внешних параметров функции фотоионизации, рекомбинации, температуры нейтралов, электронов и ионов, а также концентрации молекул $N_2$, $O_2$ и $O$ задаются аналитическими формулами. Для концентраций используем Больцмановское распределение по высоте. Распределение температур считаем экспоненциальным, что хорошо согласуется с данными наблюдений. 
Функции рекомбинации и фотоионизации (в дневное время) можно приближенно вычислять по формулам, представленным на слайде.

%\textit{Шпаргалка:} ($R=\dfrac{R_0}{M_{air}}\approx 287$~Дж$\cdot$кг$^{-1}\cdot$K$^{-1}$, $R_0$~---~универсальная газовая постоянная).

\medskip

{\bf Слайд 5: свойства решения и требования к схемам}

Уравнения, описанные в предыдущем разделе, имеют ряд особенностей, которые необходимо учитывать при численном моделировании.

В силу своего физического смысла (как концентрации электронов) решение дифференциальной задачи неотрицательно. От разностной схемы требуется сохранение неотрицательности значений $n$ на следующем временном слое, если это свойство было выполнено на предыдущем (монотонность схемы по Годунову). 


Для дифференциальной задачи имеет место закон сохранения массы, поэтому для сохранения уравнения баланса при численном моделировании используются консервативные схемы.

Важной особенностью рассматриваемой задачи является изменение входящих в уравнение коэффициентов $D, P, k, u$ на рассматриваемом отрезке на несколько порядков. 
Характерные времена различных физических процессов существенно отличаются, поэтому рассматриваемая задача жесткая.

Кроме этого рассматриваемое уравнение при неизменных $P$ и $k$ имеет стационарное решение.

\medskip

{\bf Слайд 6: метод расщепления.}


Выберем последовательно в полученном трёхмерном уравнении ключевые процессы, формирующие поле скоростей. Затем реализуем модель поэтапно, учитывая каждый раз новые поправки и сравнивая новое решение с предыдущим.

На каждом шаге по времени решаем сначала разностную задачу, отвечающую уравнению, включающему плазмохимические процессы и диффузию в $z$-проекции. Затем, взяв полученное решение в качестве начального условия, решаем задачу со следующей частью оператора. В результате после проведения первых двух шагов оказывается решено уравнение в двумерной $y-z$ постановке без учёта нейтрального и поперечного переноса. Наконец, на третьем шаге решается разностная задача, отвечающая оператору трёхмерного переноса с начальным условием~---~решением со второго шага. 

В данной работе реализован численный алгоритм для первого шага метода расщепления. Несмотря на это, уже такая приближённая постановка имеет смысл в средних широтах и вблизи полюсов, при отсутствии возмущений~---~диффузионные и плазмохимические процессы в этих областях преобладают.



\medskip

{\bf Слайд 7: первый шаг метода расщепления, три постановки}

Решение рассматриваемой приближённой задачи, отвечающей первому шагу метода расщепления, проводится в несколько этапов. В первом приближении оставим только фотохимию и амбиполярную диффузию, считая, что диффузия происходит только вдоль оси $z$. При этом получим следующее одномерное уравнение для электронной концентрации $n$.

Следующим шагом учтём широтную зависимость в уравнении. Простейший способ~---~замена коэффициента диффузии $D$ на $D\sin^2I$, где $I$~---~угол наклонения магнитных силовых линий.

Более точный учёт широтной зависимости решения приводит к двумерной задаче, включающей диффузию вдоль оси $z$ в проекции (со смешанной производной).

Отметим отдельно постановку граничных условий на верхней и нижней границах: на нижней границе в силу преобладания плазмохимических процессов ставится условие Дирихле $n = \dfrac{P}{k}$. На верхней границе поток $D\dfrac{\partial n}{\partial z} + \left(\dfrac{1}{T_p}\dfrac{\partial T_p}{\partial z} + \dfrac{1}{H}\right)n$ считается постоянным (условие 3 рода).

\medskip

{\bf Слайд 8: используемые схемы}

Перейдем к получению используемых разностных схем. 

%\textit{Шпаргалка} (Шаги по пространству: $h_i = z_{i+1}-z_i, h_{i+1/2}=z_{i+1/2}-z_{i-1/2}$.)

В точке $z=z_i$ для слагаемого $\dfrac{\partial}{\partial z}D\dfrac{\partial n}{\partial z}$ в разностных схемах используется следующая аппроксимация, полученная двойным применением формулы центральной разности на отрезках $[z_{i-1};z_i]$ и $[z_i; z_{i+1}]$

Соответственно, в численных экспериментах протестированы три различные разностные схемы: 
\begin{itemize}
\item[•] В схеме $1$ потоковый член и граничное условие аппроксимируются с помощью направленных разностей; 
\item[•] В схеме $2$ только потоковый член в уравнении записывается с помощью направленных разностей, а граничное условие всё еще использует центральные разности (несогласованное граничное условие);
\item[•] Наконец, схема $3$ имеет согласованные граничное условие и схему, записанные с помощью центральных разностей.
\end{itemize}

\smallskip

Для уравнения со смешанной производной введём обозначение $u_\varphi=-\dfrac{1}{a}D\sin I \cos I\dfrac{\partial \ln n}{\partial \varphi}=-\dfrac{1}{a}D\sin I \cos I\dfrac{1}{n}\dfrac{\partial n}{\partial \varphi}.$ Для рассматриваемого уравнения $u_\varphi$~---~это добавка к эффективной скорости, связанная со смешанной производной по $\varphi$.

\medskip

{\bf Слайды 9-10: Стационарное решение одномерной задачи}

Прежде всего обратимся к одномерной задаче для $z$-диффузии без проекций. Исследуемая задача имеет не зависящее от времени решение, при итерациях по времени происходит устсановление решения во всех трёх указанных схемах. 

Результаты расчетов (стационарные решения в зависимости от разного количества узлов по пространству) представлены на следующих графиках (по горизонтальной оси масштаб выбран логарифмическим). Соответственно, $80$ и $2000$ узлов отвечают шагам по времени $5$~км и $0{,}2$~км.

Видим, что для схемы с несогласованным граничным условием необходим гораздо более мелкий шаг, чтобы сойтись к нужному решению (возникает ложный источник на верхней границе, гр. условие ставится не по потоку в разностной схеме).

\medskip
 
{\bf Слайд 11-12: чувствительность к изменению внешних параметров}

Полученное решение позволяет исследовать чувствительность к изменению различных входящих в уравнение внешних параметров: температурам и концентрациям нейтральных молекул. При исследовании в каждом случае вычислялось стационарное решение при изменённом параметре.

Варьирование входящих в уравнение параметров показывает, что наибольшую чувствительность решение имеет к температуре нейтральных молекул, а из концентраций нейтральных молекул~---~к концентрации атомарного кислорода. 

\medskip

{\bf Слайд 13: учёт широтной зависимости}

Учтём теперь широтную зависимость в уравнении.  Установим не зависящую от времени ионизацию $P(z, t) \equiv P_0(z)$ и изучим сходимости к стационарным решениям с одних и тех же начальных условий~---~вектора с компонентами, равными единице (такой вектор отвечает <<почти нулевому>> решению). 

Полученные решения при широтах $\varphi = -30^\circ$ и $\varphi=-2^\circ$ представлены на графиках. Видно, что решения существенно различны вблизи экватора. На самом деле решения, получаемые по этим схемам вблизи экватора некорректны, необходимо уточнять их, используя полную трёхмерную модель.

\medskip

{\bf Слайд 14-15: Моделирование суточного хода в одномерной модели}

В ходе численного эксперимента по моделированию суточного хода в одномерной модели с учётом широтной зависимости вычисляется стационарное решение одномерной задачи при дневном значении $P(z)$, а затем итерации по времени продолжаются с уже меняющимся $P(z, t)$ в соответствии со введённой формулой.

Результаты представлены следующим графиком~---~трёхмерной поверхностью, построенной над плоскостью $(z, t)$.

Видно, что за сутки решение восстанавливается. Кроме того, при зенитных углах больше $90^\circ$ функция фотоионизации $P$ зануляется и начинается спад электронной концентрации, сопровождающийся изломом по времени.



\end{document}



