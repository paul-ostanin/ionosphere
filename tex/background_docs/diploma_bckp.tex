\documentclass[14pt, a4paper, fleqn]{extarticle}

\usepackage[T2A]{fontenc}
\usepackage[utf8]{inputenc}
\usepackage[english,russian]{babel}
\usepackage{url}
%\usepackage{pscyr}

%\renewcommand{\rmdefault}{ftm}
\usepackage{setspace}
\onehalfspacing

\usepackage{changepage}
\usepackage{indentfirst} %первый абзац
%%\usepackage{moreverb}
\usepackage[noend]{algorithmic}
\usepackage{amssymb, amsmath, multicol,amsthm}
%%
\usepackage{enumitem, multicol}
\usepackage{titleps,lipsum}
%%
\usepackage{mathrsfs}
\usepackage{verbatim}
\usepackage{pb-diagram}
\usepackage{graphicx}
\graphicspath{ {images/} }
\usepackage{wrapfig}
\usepackage{xcolor}
\definecolor{new}{RGB}{255,184,92}
\definecolor{news}{RGB}{112,112,112}
\usepackage{wallpaper}
\usepackage{float}
\usepackage{hyperref}
\hypersetup{
%colorlinks=true,%
%linkcolor=news,%
linkbordercolor=new,
}



\usepackage{geometry}
\geometry{top=3cm,bottom=2cm,left=2cm,right=2cm}

%\flushbottom
%\ruggedbottom

\binoppenalty=5000
\parindent=0pt

\newcommand{\EDS}{\ensuremath{\mathscr{E}}}
\newcommand*{\hm}[1]{#1\nobreak\discretionary{}%
{\hbox{$\mathsurround=0pt #1$}}{}}
\newcommand{\divisible}{\mathop{\raisebox{-2pt}{\vdots}}}
\renewcommand{\theequation}{\arabic{equation}}
\def\hm#1{#1\nobreak\discretionary{}{\hbox{$#1$}}{}}
\newcommand{\bbskip}{\bigskip \bigskip}



%%\DeclareMathOperator{\tg}{tg}
%%\DeclareMathOperator{\ctg}{ctg}

\let\leq\leqslant
\let\geq\geqslant



% Remove brackets from numbering in List of References
\makeatletter
\renewcommand{\@biblabel}[1]{\quad#1.}
\makeatother
% Header and Footer with logo
\usepackage{lastpage,fancyhdr,graphicx}
\usepackage{epstopdf}
\pagestyle{myheadings}
\pagestyle{fancy}
\fancyhf{}
\fancyfootoffset{0in}

\newpagestyle{main}{%
	\setheadrule{.4pt}%
	\sethead
		[\subsectiontitle][][\thepage]
		{\thepage}{}{\sectiontitle} }
\pagestyle{main}
\renewcommand{\headrulewidth}{0pt}  % убрать разделительную линию


\begin{document}


\tableofcontents
\clearpage

\section{Введение}
%\addcontentsline{toc}{section}{Введение}

\section{Модель ионосферы}
\sectionmark{Модель ионосферы}
%\addcontentsline{toc}{section}{Модель ионосферы}

\subsection{Вывод используемых уравнений}
\subsectionmark{Вывод используемых уравнений}

Ионосфера~---~это ионизованная часть верхней атмосферы, приблизительно от 60~км до 1000~км, целиком окружающая Землю. Основной источник плазмы~---~фотоионизация нейтральных молекул под действием солнечного ультрафиолета и рентгеновского излучения. Ионы вступают в химические реакции с нейтральными молекулами, рекомбинируют с электронами и диффундируют в другие высоты или перемещаются нейтральным ветром. Но диффузия и перенос подвержены влиянию собственного магнитного поля Земли.

\medskip

В рассматриваемом приближении моделируется эволюция концентрации $n_e$ электронов во времени и пространстве в верхней ионосфере (в F-слое).

\medskip

Основным используемым уравнением по существу является уравнение неразрывности для электронной концентрации, выражающее закон сохранения массы: $\dfrac{\partial n_e}{\partial t}+\vec{\nabla}(n_e \vec{u})=P_1-kn_e$, где $u$~---~средняя скорость диффузии. Слагаемые в правой части выражают наличие процессов образования (прямой ионизации при столкновении $O$ и $O^+$) и потерь (процессов рекомбинации).

Считаем $n_e=n_i = n(O^+)$, т.~е. рассматриваем только электроны и ионы кислорода (в F-слое больше всего ионов $O^+$), а плазму считаем квазинейтральной (в связи с этим уравнение непрерывности можно было записать в том же виде и для $n_i$ вместо $n_e$).

\medskip

За перенос в верхней атмосфере в рассматриваемой модели отвечает амбиполярная диффузия. Её суть заключается в следующем: масса электрона гораздо меньше, чем масса иона кислорода, вследствие чего электроны и ионы разделяются пространственно. Вследствие этого разделяются и заряды, и возникает электростатическое поле. Это поле препятствует дальнейшему разделению слоёв. После его создания электроны и ионы движутся как единый газ.

Для вывода уравнения амбиполярной диффузии используем гораздо более общее уравнение движения: $$nm\dfrac{D\vec{u}}{Dt}+\vec{\nabla} p + \vec{\nabla}\cdot \hat{\tau} - ne(\vec{E}+\vec{u}\times \vec{B})+nm[-\vec{G}+2\vec{\Omega}\times \vec{u}+\vec{\Omega}\times(\vec{\Omega}\times\vec{r})]=$$
$$=\sum_t nm\nu_t (\vec{u}_t-\vec{u})+ \vec{f}(\vec{q})$$
В этом уравнении $\dfrac{D}{Dt}=\dfrac{\partial}{\partial t}+(\vec{u}; \vec{\nabla})\vec{u}$~---~полная производная; $u$~---~скорость диффузии; $n$~---~концентрация электронов; $m$~---~масса электрона; $\hat{\tau}$~---~тензор напряжений; $\vec{G}$~---~ускорение свободного падения; в квадратных скобках помимо $\vec{G}$ стоят слагаемые, связанные с неинерциальностью системы отсчета, вызыванной вращением Земли с угловой скоростью $\vec{\Omega}$; в правой части сумма отвечает за столкновения электронов с различными частицами (разные типы частиц нумеруются индексом $t$), а последнее слагаемое зависит от тепловых потоков, которые в частично ионизованной плазме малы и отбрасываются. Уравнение записано в системе координат, связанной с Землёй.

\bigskip

Далее используем т.~н. диффузионную аппроксимацию: отбросим всю полную производную в силу её малости по сравнению с градиентом давления:

\begin{itemize}

\item[•] $\dfrac{|nm(\vec{u}; \vec{\nabla})\vec{u}|}{|\vec{\nabla}p|}\approx \dfrac{nmu^2/L}{nkT}\approx \dfrac{u^2}{kT/m}=M^2$~---~число Маха. При малых числах Маха (т.~е. при дозвуковых течениях) первое слагаемое можно отбросить.

\item[•] $\dfrac{|nm\partial\vec{u}/\partial t|)\vec{u}}{|\vec{\nabla}p|}\approx \dfrac{L}{\tau} \dfrac{u}{kT/m}\approx M\dfrac{L/\tau}{\sqrt{kT/m}}$, где $\tau$ и $L$~---~характерное время процессов в плазме и характерный размер области, в которой находится рассматриваемая часть плазмы.

\end{itemize}

Для медленно меняющихся и дозвуковых потоков диффузионная аппроксимация применима.

\bigskip

Помимо этого предполагаем квазинейтральность: $n_e=n_i$, движение электронов с ионами единым целым: $n_e \vec{u}_e=n_i \vec{u}_i$. Отбросим также и слагаемые, связанные с неинерциальностью системы отсчета в предположении их малости в сравнении с силами тяжести и магнитными силами.

Как уже упоминалось выше, считаем, что имеются ионы всего одного вида~---~кислорода. Это справедливо в рассматриваемом F-слое (выше $\approx 130$~км и до $1000$~км).

Запишем в приведенных приближениях общее уравнение движения для электронов и ионов в проекциях на магнитные силовые линии (индекс $\parallel$ указывает соответствующую проекцию):

$$\begin{cases}
\vec{\nabla}_\parallel p_i + (\vec{\nabla}\cdot \hat{\tau}_i)_\parallel + n_ie\vec{E}_\parallel-n_im_i\vec{G}_\parallel=\\
\textrm{ }\textrm{ }\textrm{ }\textrm{ }\textrm{ }=n_im_i\nu_{ie}(\vec{u}_e-\vec{u}_i)_\parallel+n_im_i\nu_{in}(\vec{u}_n-\vec{u}_i)_\parallel\\
\vec{\nabla}_\parallel p_e + (\vec{\nabla}\cdot \hat{\tau}_e)_\parallel - n_ee\vec{E}_\parallel-n_em_e\vec{G}_\parallel=\\
\textrm{ }\textrm{ }\textrm{ }\textrm{ }\textrm{ }=n_em_e\nu_{ei}(\vec{u}_i-\vec{u}_e)_\parallel+n_em_e\nu_{en}(\vec{u}_n-\vec{u}_e)_\parallel
\end{cases}$$

Сложив эти уравнения и учтя, что $n_e=n_i, \vec{u}_e=\vec{u}_i, n_im_i\nu_{ei}=n_em_e\nu_{ei}$, получим уравнение, не содержащее электрического поля, возникшего вследствие разделения электронов и ионов по слоям:
$$\vec{\nabla}_\parallel (p_i+p_e) + (\vec{\nabla}\cdot (\hat{\tau}_i+\hat{\tau}_e))_\parallel-n_i(m_i+m_e)\vec{G}_\parallel=n_i(m_i\nu_{in}+m_e\nu_{en})(\vec{u}_n-\vec{u}_i)_\parallel$$

Теперь можно отбросить все слагаемые, содержащие массу электрона по сравнению с такими же слагаемыми, но уже с массой иона. После этого заменим давление $p_i=n_ikT_i, p_e=n_ikT_e$ и обозначим $T_p=\dfrac{1}{2}(T_e+T_i)$. Выразив из правой части векторную разность скоростей, получим закон амбиполярной диффузии, дающий характеристику средней скорости диффундирующих ионов: 
$$\vec{u}_{i\parallel} = \vec{u}_{n\parallel} - \dfrac{2kT_p}{m_i\nu_{in}}\left(\dfrac{1}{n_i}\vec{\nabla}_\parallel n_i+\dfrac{1}{T_p}\vec{\nabla}_\parallel T_p-\dfrac{m_i\vec{G}_\parallel}{2kT_p}+\dfrac{(\vec{\nabla}\cdot\hat{\tau_i})_\parallel}{2n_ikT_p}\right)$$

Обозначим $D_a=\dfrac{2kT_p}{m_i\nu_in}$~---~коэффициент амбиполярной диффузии.

Теперь обратимся к компоненте скорости, ортогональной вектору $\vec{B}$: обозначим $\vec{E}'_\perp=\vec{E}_\perp+\vec{u}\times \vec{B}$.

\vspace{30mm}
{\bf \Large Здесь пропущена часть вывода, связанная с компонентой, перпендикулярной полю $\vec{B}$.}
\vspace{20mm}

Окончательно получаем следующее векторное уравнение, описывающее искомую эволюцию рассматриваемой электронной концентрации: $$\dfrac{\partial n_i}{\partial t} = -div(n_i \vec{u})-div\left(n_i\dfrac{1}{B^2}[\vec{E'}\times \vec{B}] \right)+$$ $$+div\left(D\left[\nabla_\parallel n_i +n_i\dfrac{1}{T_p}\nabla_\parallel T_p - \dfrac{n_i m_i}{2kT_p}\vec{g}_\parallel\right]\right)+[P-k_in_i] \eqno(2.1)$$
Объединяя три блока в правой части, можем записать это уравнение в форме $$\dfrac{\partial n_i}{\partial t} =  DYZ(n_i)+DTr(n_i)+Tr(n_i)+[P-k_in_i]. \eqno(2.2)$$

\subsection{Внешние параметры уравнения}
\subsectionmark{Внешние параметры уравнения}

Входящие в уравнение в качестве внешних параметров функции фотоионизации, рекомбинации, температуры нейтралов, электронов и ионов, а также концентрации молекул $N_2$, $O_2$ и $O$ задаются аналитическими формулами. 

Для концентраций используем Больцмановское распределение по высоте: $$n_{O_2, N_2, O} (z)= n_{O_2, N_2, O} (z_0)\cdot \exp\left(-\dfrac{M_{O_2, N_2, O}g}{R_0T_n}(z-z_0)\right).$$ Концентрации на высоте $z_0\approx 100$~км полагаем равными $n_{O_2} = 5{,}6\cdot 10^9$~см$^{-3}$, $n_{O} = 2{,}8\cdot 10^{10}$~см$^{-3}$, $n_{N_2} = 5{,}2\cdot 10^{10}$~см$^{-3}$. 

Температуры вычисляем по аналитическим формулам $$T(z)=T_\infty - (T_\infty-T_0)\exp\left(-\dfrac{g}{RT_\infty}(z-z_0)\right),$$ где $R=\dfrac{R_0}{M_{air}}\approx 287$~Дж$\cdot$кг$^{-1}\cdot$K$^{-1}$, $R_0$~---~универсальная газовая постоянная. Константы $T_\infty$ для разных составляющих приближённо считаем равными $T_{n\infty}=800$~К, $T_{i\infty}=950$~К, $T_{e\infty}=2200$~К.

Функции рекомбинации и фотоионизации (в дневное время) можно приближенно вычислять по следующим формулам: $$P=4\cdot10^{-7}n_O(z)\textrm{ (с}^{-1}\textrm{)}$$ $$k=1{,}2\cdot10^{-12}n_{N_2}(z)+2{,}1\cdot10^{-11}n_{O_2}(z) \textrm{ (с}^{-1}\textrm{)}$$

При неизменных по времени функциях фотоионизации и рекомбинации $P$ и $k$ рассматриваемые уравнения имеют стационарное решение, отвечающее по существу состоянию системы в один определённый момент времени, а вертикальный профиль соответствует фиксированным широте и долготе. Для моделирования суточного изменения вертикального профиля добавим зависимость от времени в слагаемое $P$, отвечающее фотоионизации.

Используем формулу $$P(z, t) =\begin{cases}
P_0(z)e^{\tau_0(z)(1-\sec\chi)}, |\chi|\leq\dfrac{\pi}{2}\\
0, |\chi|\geq\dfrac{\pi}{2}
\end{cases}$$

Здесь использованы следующие обозначения: $P_0(z)$~---~фотоионизация в дневное время, $\chi$~---~зенитный угол Солнца (угол между направлением на Солнце и нормалью к земной поверхности), $\tau_0(z)$~---~оптическая толщина, для вычисления которой используется формула $$\tau_0(z)=\sum_{i = N_2, O_2, O} \sigma_i^{abs}\left[\dfrac{R_0T_n}{M_i g}n_i(z)\right]= \dfrac{R_0T_n}{g}\left(\sigma_{N_2}^{abs}\dfrac{n_{N_2}(z)}{M_{N_2}}+\sigma_{O_2}^{abs}\dfrac{n_{O_2}(z)}{M_{O_2}}+\sigma_{O}^{abs}\dfrac{n_{O}(z)}{M_{O}}\right).$$

Константы $\sigma_i^{abs}$ для трёх типов нейтральных молекул известны и равны соответственно $\sigma_{N_2}^{abs}=1{,}5\cdot 10^{-17}$~см$^2$, $\sigma_{O_2}^{abs}=2\cdot 10^{-17}$~см$^2$, $\sigma_{O}^{abs}=1\cdot 10^{-17}$~см$^2$.

Характерные величины оптической толщины на различных высотах представлены в следующей таблице:

\smallskip

$$\begin{tabular}{|c|c|c|c|}
\hline
&$z_1=100$~км&$z_2=300$~км&$z_3=500$~км\\
\hline
$\tau_0$&$4\cdot 10^2$&$3\cdot 10^{-1}$&$2\cdot 10^{-4}$\\
\hline
\end{tabular}$$

\medskip


В предложенной формуле для фотоионизации время в качестве параметра входит лишь в зенитный угол. Кусочное задание функции $P(z, t)$ связано с приближением отсутствия фотоионизации в ночное время (Солнце не заходит за горизонт лишь при зенитных углах, не превосходящих $90^\circ$).

Зависимость зенитного угла от времени даётся следующими формулами: $$\cos\chi = \sin\varphi\cdot\sin\delta-\cos\varphi\cdot\cos\delta\cdot\cos\omega t$$

Здесь $\omega$~---~угловая скорость вращения Земли, $\varphi$~---~широта, а $\delta$~---~склонение Солнца, тангенс которого определяется формулой $$\tg\delta = \tg 23{,}5^\circ \cdot \sin\left(2\pi\cdot\dfrac{d-80}{365}\right),$$ где $d$~---~номер дня от начала года. 


\subsection{Расщепление по физическим процессам}
\subsectionmark{Расщепление по физическим процессам}

Выберем последовательно в полученном трёхмерном уравнении ключевые процессы, формирующие поле скоростей. Затем реализуем модель поэтапно, учитывая каждый раз новые поправки и сравнивая новое решение с предыдущим.

В первом приближении из трёх введённых слагаемых в дивергентном виде оставим только первое, считая, что диффузия происходит только вдоль оси $z$ (поле считается вертикальным). При этом получим следующую одномерную задачу для электронной концентрации $n$:
$$\begin{cases}
\dfrac{\partial n}{\partial t} = P-kn+\dfrac{\partial}{\partial z}\left(D\dfrac{\partial n}{\partial z} + u n\right)\\
n|_{z=100\mbox{ }km} = \dfrac{P(z=100\mbox{ }km)}{k(z=100\mbox{ }km)}\\
\left(D\dfrac{\partial n}{\partial z} + u n\right)\bigg|_{z=500\mbox{ }km} = F=const
\end{cases} \eqno(2.3)
$$
Здесь $D$~---~коэффициент амбиполярной диффузии, $u = D\left(\dfrac{1}{T_p}\dfrac{\partial T_p}{\partial z}+\dfrac{m_ig}{2kT_p}\right)$~---~эффективная скорость, $P$ и $kn$~---~слагаемые, отвечающие процессам ионизации при столкновении $O$ и $O+$ и рекомбинации соответственно. В используемой модели температура, концентрация нейтралов, зависимости $D(z), k(z)$ и $P_1(z)$~---~внешние параметры.

\bigskip

Следующим шагом учтём широтную зависимость в уравнении. Простейший способ~---~замена коэффициента диффузии $D$ на $D\sin^2I$, где $I$~---~угол наклонения магнитных силовых линий, $I \approx \arctg(2\tg \varphi)$, $\varphi$~---~широта ($\varphi \in [-90^\circ; +90^\circ]$). При этом уравнение заменится на следующее:
$$\dfrac{\partial n}{\partial t} =P-kn+\dfrac{\partial}{\partial z}\left[\sin^2I\left(D\dfrac{\partial n}{\partial z} + \left(\dfrac{1}{T_p}\dfrac{\partial T_p}{\partial z}+\dfrac{1}{H}\right) n\right)\right] \eqno(2.4)$$

В рассматриваемой постановке широта $\varphi$~---~внешний задаваемый параметр. При фиксированном $\varphi$ уравнение, как и предыдущее, имеет стационарное решение.

Особенность данной постановки состоит в том, что на экваторе при $\varphi=0$ уравнение вырождается: ненулевыми остаются только производная по времени в левой части и $P-kn$ в правой. Этот эффект не соответствует никакому физическому явлению, уравнение не описывает физические процессы на экваторе.

Более точный учёт широтной зависимости решения приводит к двумерной задаче, включающей диффузию вдоль оси $z$ в проекции (со смешанной производной):
$$\dfrac{\partial n}{\partial t} =P-kn+\dfrac{\partial}{\partial z}\biggl[D\sin^2 I\left(\dfrac{\partial n}{\partial z}+\left(\dfrac{1}{T_p}\dfrac{\partial T_p}{\partial z}+\dfrac{1}{H}\right)n\right)-$$ $$-\dfrac{1}{a}D\sin I\cos I\left(\dfrac{\partial n}{\partial\varphi}+\dfrac{1}{T_p}\dfrac{\partial T_p}{\partial\varphi}n\right)\biggr] \eqno(2.5)$$


\newpage

\section{Численное моделирование}
\sectionmark{Численное моделирование}

\subsection{Одномерное уравнение}
\subsectionmark{Одномерное уравнение}

Уравнения, описанные в предыдущем разделе, имеют ряд особенностей, которые необходимо учитывать при численном моделировании. Для начала рассмотрим одномерное уравнение для $z$-диффузии без проекций $(2.3)$.

Вблизи нижней границы влияние диффузионного слагаемого и переноса пренебрежимо малы по сравнению с процессами фотохимии. Напротив, на верхней части исследуемого высотного интервала преобладают диффузионные процессы, а $P$ и $k$ уже не играют роли. Важной особенностью рассматриваемой задачи является изменение входящих в уравнение коэффициентов $D, P, k, u$ на рассматриваемом отрезке на несколько порядков. Характерные величины на нескольких высотах представлены в следующей таблице: 

\smallskip

\begin{tabular}{|c|c|c|c|}
\hline
&$z_1=200$~км&$z_2=300$~км&$z_3=500$~км\\
\hline
$D$, см$^{2}\cdot$с$^{-1}$&$3{,}1\cdot 10^9$&$3{,}4\cdot 10^{10}$&$4{,}2\cdot 10^{12}$\\
\hline
$k$, с$^{-1}$&$5{,}2\cdot 10^{-3}$&$5{,}5\cdot 10^{-5}$&$1{,}3\cdot 10^{-8}$\\
\hline
$P_1$, см$^{-3}\cdot$с$^{-1}$&$1{,}5\cdot 10^3$&$1{,}2\cdot 10^{2}$&$1{,}3$\\
\hline
$u_\textrm{эфф}/D$, см$^{-1}$&$4{,}8\cdot 10^{-8}$&$4{,}5\cdot 10^{-8}$&$3{,}6\cdot 10^{-8}$\\
\hline
\end{tabular}

\medskip 

Характерные времена различных физических процессов существенно различны, поэтому рассматриваемая задача жесткая. Следовательно, по времени рассматриваем неявные схемы: во всех случаях производную по времени аппроксимируем по формуле $\dfrac{\partial n}{\partial t}\approx \dfrac{n^{j+1}-n^j}{\tau}$, а в правой части все слагаемые берём на следующем временном слое с номером $(j+1)$. С учётом этого замечания далее в записи различных аппроксимаций правой части будем писать только нижние индексы у $n$, подразумевая всегда верхний индекс $(j+1)$.

От разностной схемы требуется выполнение закона сохранения массы, а также сохранение неотрицательности значений $n$ на следующем временном слое, если это свойство было выполнено на предыдущем. Эти требования связаны с отсутствием физического смысла у решений, не сохраняющих массу или содержащих отрицательные значения концентрации.

Перейдем к получению используемых разностных схем. Введём следующие обозначения для шагов по пространству: $$h_i = z_{i+1}-z_i$$ $$h_{i+1/2}=z_{i+1/2}-z_{i-1/2}$$
В точке $z=z_i$ для слагаемого $\dfrac{\partial}{\partial z}D\dfrac{\partial n}{\partial z}$ в разностных схемах используется следующая аппроксимация, полученная двойным применением формулы центральной разности на отрезках $[z_{i-1};z_i]$ и $[z_i; z_{i+1}]$: 
$$\dfrac{\partial}{\partial z}D\dfrac{\partial n}{\partial z} \approx \dfrac{1}{h_{i+1/2}}\left(\dfrac{D_{i+1/2}(n_{i+1}-n_i)}{h_i}-\dfrac{D_{i-1/2}(n_{i}-n_{i-1})}{h_{i-1}}\right)$$
Для слагаемого $\dfrac{\partial}{\partial z}(nu)$ исследуем схемы направленных разностей $\dfrac{u_{i+1}n_{i+1}-u_{i}n_{i}}{h_i}$, а также центральных разностей $\dfrac{u_{i+1}n_{i+1}-u_{i-1}n_{i-1}}{h_{i-1}+h_{i+1}}$.

Нижнее граничное условие (условие Дирихле) аппроксимируется точно, а на верхней границе условие постоянства потока может быть записано несколькими способами. Для данной одномерной задачи используем две различных аппроксимации этого условия:

\begin{itemize}
\item[•] В первом случае поток $\dfrac{\partial n}{\partial z}+\dfrac{u_N}{D_N}\cdot n_N=F$ аппроксимируется с помощью центральных разностей по пространству, что соответствует схеме $$n_N-n_{N-1}+u_N/D_N\cdot h_N\cdot n_N = F\cdot h_N$$
\item[•] Во втором случае для схемы центральных разностей запишем согласованную схему для верхнего граничного случая, получаемую с помощью интегрирования уравнения на $N$-ом шаге по пространству между двумя соседними полуцелыми узлами, а также учёта равенства потока на верхнем полуцелом узле заданной величине $F$: $$h_{N+1/2}\dfrac{n^{j+1}-n^j}{\tau}= F - D_{N-1/2}\dfrac{n_N-n_{N-1}}{h_{N-1}}-\dfrac{1}{2}(u_{N-1}n_{N-1}^{j+1}+u_{N}n_{N}^{j+1})$$
\end{itemize}

Соответственно, в численных экспериментах протестированы три различные разностные схемы: 
\begin{itemize}
\item[•] В схеме $1$ потоковый член и граничное условие аппроксимируются с помощью центральных разностей; 
\item[•] В схеме $2$ только потоковый член в уравнении записывается с помощью центральных разностей, а граничное условие всё еще использует центральные разности;
\item[•] Наконец, схема $3$ имеет согласованные граничное условие и схему, записанные с помощью центральных разностей.
\end{itemize}

Для одномерного уравнения $(2.4)$ с учётом проекции на магнитную силовую линию (с помощью добавления множителя $sin^2 I$) использованы те же разностные схемы, но с добавлением широты $\varphi$ в качестве внешнего параметра.



\subsection{Квазидвумерная постановка}
\subsectionmark{Квазидвумерная постановка}

Рассмотрим теперь разностные схемы для уравнения, описывающего $z$-диффузию в проекции с добавлением смешанной производной $(2.5)$:$$\dfrac{\partial n}{\partial t} =P-kn+\dfrac{\partial}{\partial z}\biggl[D\sin^2 I\left(\dfrac{\partial n}{\partial z}+\left(\dfrac{1}{T_p}\dfrac{\partial T_p}{\partial z}+\dfrac{1}{H}\right)n\right)-$$ $$-\dfrac{1}{a}D\sin I\cos I\left(\dfrac{\partial n}{\partial\varphi}+\dfrac{1}{T_p}\dfrac{\partial T_p}{\partial\varphi}n\right)\biggr]$$

Заменим $u = \dfrac{1}{T_p}\dfrac{\partial T_p}{\partial z}+\dfrac{1}{H}$. Для использования уже имеющегося программного кода в применении уже к данной двумерной задаче используем следующую разностную схему: для смешанной производной $\dfrac{\partial}{\partial z}\dfrac{\partial n}{\partial \varphi}$ запишем $$\dfrac{\partial}{\partial z}\dfrac{\partial n}{\partial \varphi}=\dfrac{\partial}{\partial z}\left(n\dfrac{1}{n}\dfrac{\partial n}{\partial \varphi}\right) = \dfrac{\partial}{\partial z}\left(n\dfrac{\partial \ln n}{\partial \varphi}\right)$$

Введём обозначение $u_\varphi=-\dfrac{1}{a}D\sin I \cos I\dfrac{\partial \ln n}{\partial \varphi}=-\dfrac{1}{a}D\sin I \cos I\dfrac{1}{n}\dfrac{\partial n}{\partial \varphi}.$ Для рассматриваемого уравнения $u_\varphi$~---~это добавка к эффективной скорости, связанная со смешанной производной по $\varphi$.

Используем для численного решения нелинейную схему: будем вычислять решение, последовательно перемещаясь по временным слоям, причем $u_\varphi$ будем брать на основании данных с предыдущего временного слоя, а $n$~---~со следующего.

Для рассматриваемого уравнения в результате применяется та же разностная схема с центральными разностями, что и для одномерного уравнения, но возникает добавка, связанная с последним слагаемым. Для этого слагаемого использованы две различные разностные аппроксимации:
\begin{itemize}
\item[•] Схема направленных разностей с учётом возможной знакопеременности эффективной скорости: $\dfrac{|u_\varphi|+u_\varphi}{2}\cdot\dfrac{n_{i+1}-n_i}{h_i} + \dfrac{|u_\varphi|-u_\varphi}{2}\cdot\dfrac{n_{i-1}-n_i}{h_{i-1}}$
\item[•] Схема центральных разностей: $\dfrac{(u_\varphi)_{i+1}n_{i+1}-(u_\varphi)_{i-1}n_{i-1}}{h_{i-1}+h_{i+1}}$
\end{itemize}
В обоих случаях концентрации $n$ берутся со следующего временного слоя~---~схема неявная.

\smallskip

При этом сама эффективная скорость может быть вычислена двумя способами: 
\begin{itemize}
\item[•] Первый способ~---~применение формулы центральной разности к производной по $\varphi$ для логарифма в формуле $u_\varphi$ $$u_\varphi\approx \dfrac{\ln n^j_i(\varphi+\Delta\varphi)-\ln n^j_i(\varphi-\Delta\varphi)}{2\Delta\varphi};$$ 
\item[•] Второй способ~---~без привлечения логарифма использовать формулу $$u_\varphi \approx \dfrac{2}{n_i^j(\varphi+\Delta\varphi)+n_i^j(\varphi-\Delta\varphi)}\cdot\dfrac{n_i^j(\varphi+\Delta\varphi)-n_i^j(\varphi-\Delta\varphi)}{2\Delta\varphi}.$$
\end{itemize}

Численные эксперименты показали, что обе формулы для $u_\varphi$ дают один и тот же результат. Более того, на практике решение никогда не достигает чистого нуля, поэтому отдельные кусочные задания формул для $u_\varphi$ при нулевых значениях $n$ на предыдущем временном слое никак не отражаются на получаемом решении. Тем не менее, вторая формула более удобна для анализа асимптотического поведения $u_\varphi$ при приближении $n$ к нулю.

Для вычисления решения на следующем временном слое используются граничные условия как по $z$, так и по $\varphi$: на полюсах решение полагается тождественно равным стационарному решению одномерной задачи. Это позволяет сохранить непрерывность решения в зависимости от $\varphi \in [-90^\circ; 90^\circ]$. На нижней границе по $z$ ставится граничное условие типа Дирихле~---~решение совпадает с отношением $\dfrac{P(100\mbox{ km}, t)}{k(100\mbox{ km})}$. Верхнее граничное условие, как и ранее,~---~постоянство полного потока (с учётом добавки к эффективной скорости в виде $u_\varphi$). В разностной аппроксимации верхнего граничного условия используется центральная разность, в результате чего эта аппроксимация оказывается согласованной со схемой центральных разностей для рассматриваемого уравнения.


\section{Результаты численных экспериментов}
\sectionmark{Результаты численных экспериментов}

\subsection{Воспроизведение дневного вертикального профиля электронной концентрации}
\subsectionmark{Воспроизведение дневного вертикального профиля электронной концентрации}

Прежде всего обратимся к одномерной задаче для $z$-диффузии без проекций. Исследуемая задача имеет не зависящее от времени решение, а численные эксперименты показали, что при итерациях по времени происходит установление решения во всех трёх схемах, указанных в разделе $(3.1)$. Используемый шаг по пространству $h = 5$~км и по времени $\tau = 3$~мин обеспечивает сходимость к одной и той же кривой в схемах $1$ и $3$ с характерным временем установления порядка $4-5$ часов (по прошествии этого времени первые несколько значащих цифр в решении уже не изменяются). Схема $2$ также имеет сходимость к стационарному решению, но в отличие от оставшихся двух схем при шаг по пространству $h=5$~км слишком велик, для получения того же самого решения, что и в других двух схемах, необходимо уменьшить шаг хотя бы до $h = 0{,}2$~км.

Результаты расчетов (стационарные решения в зависимости от разного количества узлов по пространству) представлены на следующих графиках (по горизонтальной оси масштаб выбран логарифмическим). Соответственно, $80$, $400$ и $2000$ узлов отвечают шагам по времени $5$~км, $1$~км и $0{,}2$~км.
 
\begin{figure}[H]
\center{
\includegraphics[scale=0.5]{1d_stationary_logscale_80}}
\caption{Стационарные решения на $80$ расчётных узлах.}
\end{figure}

\begin{figure}[H]
\center{
\includegraphics[scale=0.5]{1d_stationary_logscale_400}}
\caption{Стационарные решения на $400$ расчётных узлах.}
\end{figure}

\begin{figure}[H]
\center{
\includegraphics[scale=0.5]{1d_stationary_logscale_2000}}
\caption{Стационарные решения на $2000$ расчётных узлах.}
\end{figure}

\subsection{Чувствительности ко внешним параметрам уравнения}
\subsectionmark{Чувствительности ко внешним параметрам уравнения}



Полученное решение позволяет исследовать чувствительность к изменению различных входящих в уравнение внешних параметров: температурам, концентрациям нейтральных молекул, фотоионизации и рекомбинации. На следующих ниже графиках представлены результаты варьирования каждого из параметров в отдельности на $10\%$ и $20\%$ (в обе стороны). В каждом случае вычислено стационарное решение при изменённом параметре, на всех графиках средняя кривая отвечает невозмущенному уравнению.

\medskip

Варьирование входящих в уравнение температур показывает, что наибольшую чувствительность решение имеет к температуре нейтральных молекул. Изменение концентрации нейтральных молекул~---~атомарного кислорода, молекулярного кислорода и азота показывает, что наибольшая чувствительность решения отвечает изменению концентрации атомарного кислорода, а чувствительности к изменению концентраций атомарного кислорода и азота приблизительно одинаковы.

\medskip

\begin{figure}
\center{
\includegraphics[scale=0.5]{Ti-sensitivity_log}}
\caption{Чувствительность к изменению температуры ионов.}
\end{figure}

\begin{figure}
\center{
\includegraphics[scale=0.5]{Te-sensitivity_log}}
\caption{Чувствительность к изменению температуры электронов.}
\end{figure}

\begin{figure}
\center{
\includegraphics[scale=0.5]{Tn-sensitivity_log}}
\caption{Чувствительность к изменению температуры нейтральных молекул.}
\end{figure}

\begin{figure}
\center{
\includegraphics[scale=0.5]{nO-sensitivity_log}}
\caption{Чувствительность к изменению концентрации атомарного кислорода.}
\end{figure}

\begin{figure}
\center{
\includegraphics[scale=0.5]{nO2-sensitivity_log}}
\caption{Чувствительность к изменению концентрации молекулярного кислорода.}
\end{figure}

\begin{figure}
\center{
\includegraphics[scale=0.5]{nN2-sensitivity_log}}
\caption{Чувствительность к изменению концентрации азота.}
\end{figure}

\newpage

\subsection{Учёт проекции на магнитную силовую линию в одномерной задаче}
\subsectionmark{Учёт проекции на магнитную силовую линию в одномерной задаче}


Учтём теперь широтную зависимость в уравнении. В качестве первого шага продолжим использование одномерного уравнения, но заменим коэффициент диффузии $D$ на $D\sin^2I$, где $I\approx \arctg(2\tg \varphi)$, $\varphi$~---~широта.

Результаты расчётов суточного хода при $\varphi = -88^\circ, -66^\circ, -30^\circ, -1^\circ$ приведены на следующих графиках:

\begin{figure}[H]
\center{
\includegraphics[scale=0.3]{diurnal_projection_-88}}
\caption{Суточный ход в одномерной модели с учётом проекции на магнитную силовую линию, $\varphi = -88^\circ$.}
\end{figure}

\begin{figure}[H]
\center{
\includegraphics[scale=0.3]{diurnal_projection_-66}}
\caption{Суточный ход в одномерной модели с учётом проекции на магнитную силовую линию, $\varphi = -66^\circ$.}
\end{figure}

\begin{figure}[H]
\center{
\includegraphics[scale=0.3]{diurnal_projection_-30}}
\caption{Суточный ход в одномерной модели с учётом проекции на магнитную силовую линию, $\varphi = -30^\circ$.}
\end{figure}

\begin{figure}[H]
\center{
\includegraphics[scale=0.3]{diurnal_projection_-2}}
\caption{Суточный ход в одномерной модели с учётом проекции на магнитную силовую линию, $\varphi = -2^\circ$.}
\end{figure}



\subsection{Согласованность аппроксимаций граничного условия и уравнения}
\subsectionmark{Согласованность аппроксимаций граничного условия и уравнения}

При численном решении квазидвумерного уравнения в предыдущих экспериментах были использованы центральные разности для аппроксимации как уравнения, так и граничного условия. Выясним, каково влияние согласованности этих аппроксимаций на решение: для верхнего граничного условия продолжим использовать центральную разность, а уравнение аппроксимируем, используя для дополнительного слагаемого со смешанной производной аппроксимацию направленными разностями (с учетом знакопеременности скорости): $$\dfrac{|u_\varphi|+u_\varphi}{2}\cdot\dfrac{n_{i+1}-n_i}{h_i} + \dfrac{|u_\varphi|-u_\varphi}{2}\cdot\dfrac{n_{i-1}-n_i}{h_{i-1}}.$$

Результат численного эксперимента показывает, что при широтах, далёких от нулевой, влияние несущественно, но вблизи экватора решение меняется заметно: 

\begin{figure}[H]
\center{
\includegraphics[scale=0.3]{diurnal_2d_inconsistent_boundary_-2}}
\caption{Использование несогласованных схем для гр. условия и для уравнения, $\varphi = -2^\circ$.}
\end{figure}


\subsection{Сравнение стационарных решений в различных постановках}
\subsectionmark{Сравнение стационарных решений в различных постановках}


Различные постановки используются для учёта наклонения магнитных силовых линий. Для исследования и сравнения качественных отличий полученных результатов установим не зависящую от времени ионизацию $P(z, t) \equiv P_0(z)$ и изучим сходимости к стационарным решениям с одних и тех же начальных условий~---~вектора с компонентами, равными единице (такой вектор отвечает <<почти нулевому>> решению, при этом даёт возможность использовать схемы, применение которых к случаю нулевых значений затруднено, как, например, в случае логарифма для $u_\varphi$). 

Полученные решения при широтах $\varphi = -66^\circ$ и $\varphi = -88^\circ$ практически не отличаются (кривые на графике совпадают), а при $\varphi = -30^\circ$ и в особенности~---~при $\varphi=-2^\circ$ различия существенны:

\begin{figure}[H]
\center{
\includegraphics[scale=0.6]{stationary_-30}}
\caption{Стационарные решения для трёх различных постановок, $\varphi = -30^\circ$.}
\end{figure}


\begin{figure}[H]
\center{
\includegraphics[scale=0.6]{stationary_-2}}
\caption{Стационарные решения для трёх различных постановок, $\varphi = -2^\circ$.}
\end{figure}



\subsection{Моделирование суточного хода в одномерной модели}
\subsectionmark{Моделирование суточного хода в одномерной модели}


В ходе численного эксперимента по моделированию суточного хода в одномерной модели вычисляется стационарное решение одномерной задачи при дневном значении $P(z)$, а затем итерации по времени продолжаются с уже меняющимся $P(z, t)$ в соответствии со введённой формулой.

Результаты представлены следующим графиком~---~трёхмерной поверхностью, построенной над плоскостью $(z, t)$.

Видно, что за сутки решение восстанавливается. Кроме того, после зануления $P$ при зенитных углах больше $90^\circ$ начинается спад электронной концентрации, сопровождающийся изломом по времени (в соответствующий  момент $P$, входящее в уравнение, также терпит излом).

Отметим также, что при обнулении $P$ решение (начиная со стационарного) падает почти до нуля приблизительно за $6$ часов.

\begin{figure}[H]
\center{
\includegraphics[scale=0.3]{diurnal_1d}}
\caption{Суточный ход в одномерной модели с добавлением зависимости фотоионизации от зенитного угла.}
\end{figure}

\subsection{Суточный ход в квазидвумерной постановке вблизи экватора}
\subsectionmark{Суточный ход в квазидвумерной постановке вблизи экватора}


На следующих графиках представлен суточный ход при тех же широтах, что и в случае одномерного уравнения с $z$-проекцией. Существенное отличие наблюдается в характере решения вблизи экватора~---~уравнение не вырождается.


\begin{figure}[H]
\center{
\includegraphics[scale=0.3]{diurnal_2d_-30}}
\caption{Суточный ход в квазидвумерной модели, $\varphi = -30^\circ$.}
\end{figure}

\begin{figure}[H]
\center{
\includegraphics[scale=0.3]{diurnal_2d_-2}}
\caption{Суточный ход в одномерной модели с учётом проекции на магнитную силовую линию, $\varphi = -2^\circ$.}
\end{figure}


\end{document}



