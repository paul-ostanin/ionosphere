\documentclass[12pt, a5paper, fleqn, twoside]{article}
\usepackage[top=20.5mm,left=12.5mm]{geometry}
\usepackage[T2A]{fontenc}
\usepackage[utf8]{inputenc}
\usepackage[english,russian]{babel}
\usepackage{url}


\usepackage{changepage}
\usepackage{indentfirst} %первый абзац
%%\usepackage{moreverb}
\usepackage[noend]{algorithmic}
\usepackage{amssymb, amsmath, multicol,amsthm}
%%
\usepackage{enumitem, multicol}
\usepackage{titleps,lipsum}
%%
\usepackage{mathrsfs}
\usepackage{verbatim}
\usepackage{pb-diagram}
\usepackage{graphicx}
\graphicspath{ {images/} }
\usepackage{wrapfig}
\usepackage{xcolor}
\definecolor{new}{RGB}{255,184,92}
\definecolor{news}{RGB}{112,112,112}
\usepackage{wallpaper}
\usepackage{hyperref}
\hypersetup{
%colorlinks=true,%
%linkcolor=news,%
linkbordercolor=new,
}



%\oddsidemargin=-12.9mm
\evensidemargin=-12.9mm
\textwidth=123mm
%\headheight=10mm
%\topmargin=-22.9mm
\textheight=175mm
\tolerance=200
\pagestyle{plain}

%\flushbottom
%\ruggedbottom

\binoppenalty=5000
\parindent=0pt

\newcommand{\EDS}{\ensuremath{\mathscr{E}}}
\newcommand*{\hm}[1]{#1\nobreak\discretionary{}%
{\hbox{$\mathsurround=0pt #1$}}{}}
\newcommand{\divisible}{\mathop{\raisebox{-2pt}{\vdots}}}
\RequirePackage{caption2}
\renewcommand\captionlabeldelim{}
\renewcommand{\theequation}{\arabic{equation}}
\def\hm#1{#1\nobreak\discretionary{}{\hbox{$#1$}}{}}
\newcommand{\bbskip}{\bigskip \bigskip}

\newtheorem{Lemma}{Лемма}
\theoremstyle{definiton}
\newtheorem{Remark}{Замечание}
%%\newtheorem{Def}{Определение}
\newtheorem{Claim}{Утверждение}
\newtheorem{Cor}{Следствие}
\newtheorem{Theorem}{Теорема}
\newtheorem{Th}{Теорема}
\newtheorem{Example}{Пример}
\newtheorem*{known}{Теорема}
\def\proofname{Доказательство}
\theoremstyle{definition}
\newtheorem{Def}{Определение}
\newtheorem{De}{Определение}
%% \newenvironment{Example} % имя окружения
%% {\par\noindent{\bf Пример.}} % команды для \begin
%% {\hfill$\scriptstyle\qed$} % команды для \end



%%\DeclareMathOperator{\tg}{tg}
%%\DeclareMathOperator{\ctg}{ctg}

\let\leq\leqslant
\let\geq\geqslant



% Remove brackets from numbering in List of References
\makeatletter
\renewcommand{\@biblabel}[1]{\quad#1.}
\makeatother
% Header and Footer with logo
\usepackage{lastpage,fancyhdr,graphicx}
\usepackage{epstopdf}
\pagestyle{myheadings}
\pagestyle{fancy}
\fancyhf{}
\fancyfootoffset{0in}

%\fancyhead[LE, RO]{Летняя Школа <<Phystech.International>>}
%\fancyhead[LO, RE]{\thepage}

\newpagestyle{main}{%
	\setheadrule{.4pt}%
	\sethead
		[\subsectiontitle][][\thepage]
		{\thepage}{}{\sectiontitle} }
\pagestyle{main}

%\renewcommand{\footrule}{\hrule height 0.5pt \vspace{2mm}}


\begin{document}

$$\begin{cases}
-\nabla p_i + n_im_i\vec{g}+en_i(\vec{E}+\vec{v_i}\times\vec{B})-n_im_i\nu_i(\vec{v_i}-\vec{u})=0\\
-\nabla p_e + n_em_e\vec{g}-en_e(\vec{E}+\vec{v_e}\times\vec{B})-n_em_e\nu_e(\vec{v_e}-\vec{u})=0 
\end{cases}$$

Обозначим $$\vec{f} = e\vec{E}+m_i\nu_i \vec{u}-\dfrac{1}{n_i}\nabla p_i+m_i \vec{g}$$

Уравнение для ионов: $$\vec{f}+e\vec{v_i}\times\vec{B}-m_i\nu_i\vec{v_i}=0 \eqno(1)$$

Выражаем $\vec{v_i}$: $$\vec{v_i}=\dfrac{\vec{f}}{m_i\nu_i}+\dfrac{e}{m_i\nu_i}\vec{v_i}\times\vec{B} \eqno(2)$$

Множим $(2)$ на $\vec{B}$ векторно и скалярно: 

$$\begin{cases}
\vec{v_i}\times \vec{B} = \dfrac{1}{m_i\nu_i}[\vec{f}\times \vec{B}-eB^2\vec{v_i}+e(\vec{v_i};\vec{B})\vec{B}]\\
(\vec{v_i}; \vec{B})= \dfrac{(\vec{f};\vec{B})}{m_i\nu_i}
\end{cases}$$

Подставим найденное скалярное произведение в правую часть выражения для векторного произведения:
$$\vec{v_i}\times \vec{B} = \dfrac{1}{m_i\nu_i}\left[\vec{f}\times \vec{B}-eB^2\vec{v_i}+e\dfrac{(\vec{f};\vec{B})}{m_i\nu_i}\vec{B}\right] $$

Теперь это векторное произведение подставим в уравнение движения $(1)$:
$$\vec{v_i} = \dfrac{\vec{f}}{m_i\nu_i}+\dfrac{e}{(m_i\nu_i)^2}\left(\vec{f}\times\vec{B}-eB^2\vec{v_i}+e\dfrac{(\vec{f};\vec{B})}{m_i\nu_i}\vec{B}\right)\eqno(3)$$

Подставим $\vec{f}$: предполагаем, что компоненты сил тяжести и градиента давления, ортогональные магнитному полю, малы по сравнению с силой Лоренца $$\left(1+\left(\dfrac{eB}{m_i\nu_i}\right)^2\right)\vec{v_i}=\dfrac{e\vec{E}}{m_i\nu_i}+\vec{u}-\dfrac{\nabla p_i}{n_im_i\nu_i}+\dfrac{\vec{g}}{\nu_i}+\dfrac{(m_i\nu_i\vec{u}-\frac{1}{n_i}\nabla p_i + m_i \vec{g};\vec{B})}{m_i\nu_i}\vec{B}$$

\end{document}





